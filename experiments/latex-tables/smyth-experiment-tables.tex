\documentclass[acmsmall,nonacm]{acmart}

\AtBeginDocument{%
  \providecommand\BibTeX{{%
    \normalfont B\kern-0.5em{\scshape i\kern-0.25em b}\kern-0.8em\TeX}}}

\usepackage{pifont}% http://ctan.org/pkg/pifont
\newcommand{\cmark}{\ding{51}}%
\newcommand{\xmark}{\ding{55}}%

\newcommand{\highlightBlue}[1]{\textcolor{blue}{\textbf{#1}}}
\newcommand{\highlightRed}[1]{\textcolor{red}{\textbf{#1}}}

\newcommand{\snsMyth}
  {\ensuremath{\textsc{Smyth}}}
\newcommand{\myth}
  {\ensuremath{\textsc{Myth}}}
\newcommand{\leon}
  {\ensuremath{\textsc{Leon}}}
\newcommand{\synquid}
  {\ensuremath{\textsc{Synquid}}}

\newcommand{\experimentTableSize}
  {\scriptsize}
\newcommand{\experimentCaptionSize}
  {\footnotesize}

\newcommand{\vsepBeforeCaption}{\vspace{0.05in}}

\begin{document}

\title{Smyth Experiment Tables}
\subtitle{ICFP 2020 Artifact Evaluation (May 22, 2020)}
\maketitle

\textbf{Changes to \snsMyth{}.}
%
Since the submission, we have made two algorithmic improvements:

\begin{enumerate}

\item Change to search order: trying to synthesize case expressions before
constructors (still subject to the same staging parameters). This change tends
to improve performance and produce more readable solutions for some benchmarks.

\item From Section 6.3: ``Future work should consider how to automatically
reconfigure staging parameters to account for the structure of user-provided
sketches.'' We have implemented such ``responsive'' staging parameters.

\end{enumerate}

We have also since removed the client-side portion of our implementation that
had been written in Elm. The new (command-line) interface is written in OCaml.

\vspace{0.30in}
\noindent
\textbf{Changes to \synquid{}.}
%
From Section 6.4: ``Discussion with the authors of \synquid{} has revealed an
implementation issue involving the axiomatization of recursive
datatypes in the underlying logic.
%
As a result, desired solutions for many benchmarks---even non-recursive
ones---failed to typecheck.
%
When this issue is addressed, \synquid{} may very well synthesize many
of these tasks (conservatively marked \textbf{?}).''

Nadia Polikarpova has recently fixed this issue
%
(\url{https://github.com/nadia-polikarpova/synquid/commit/20e6d62e151e314cc0e5b36983fded12e6e6c8c4}),
%
so we re-ran the \synquid{} experiments.


\vspace{0.30in}
\noindent
\textbf{Updated Figure 10 Results.}
%
As such, our Figure 10 results have changed slightly.
%
The following page replicates Figure 10 from our submission.
%
The rest of the document includes Tables 1, 2, 3, and 4, corresponding to the four
experiments, and describes the differences.

Tables 1, 2, and 3 include a \textit{Revised Artifact (Ours)} column to show our
updated results, using blue highlights to emphasize differences compared to the
\textit{Submission}.
%
The \textit{Revised Artifact (Yours)} column shows the results that you obtained
on your machine; differences from ours are highlighted in red.
%
If you have not run the experiments, the column should be filled with dots.


\vspace{0.30in}
%
\noindent
%
\textbf{\leon{} and \synquid{} Benchmarks (\texttt{experiments/exp-4-logic/}):}
%
For Experiment 4, we wrote a script
%
(\texttt{generate-benchmarks.py})
%
to generate \leon{} and \synquid{} tasks
%
(\texttt{generated/}),
%
which we copied and pasted into the respective web editors.
%
Our labeled results are in \texttt{results/}.
%
The helper script \texttt{show-results.sh} summarizes the labels,
and also includes comments with some shell commands we found useful during
this experiment.
%
The sketching benchmarks we tried (discussed on lines 1033-1034 of the
submission) can be found in \texttt{sketches/}.

\leon{}: If \url{https://leon.epfl.ch/} doesn't work, you might try
\url{https://leon.epfl.ch/}, a URL that Viktor Kuncak provided. In either case,
look for the ``Synthesis'' menu on the right and click ``Search.''

\synquid{}: The web editor \url{http://comcom.csail.mit.edu/comcom/#Synquid}
currently server the version with Nadia's recent changes. Nadia suggested
running our benchmarks with the \texttt{-e=True} flag; we found that
\texttt{-e=False} was needed for a few benchmarks to succeed.

\clearpage

\setcounter{figure}{9}

\newcommand{\numBenchmarks}{39}
\newcommand{\numBenchmarksAll}{43}
\newcommand{\numBenchmarksRecursive}{28}
\newcommand{\numBenchmarksBase}{24}
\newcommand{\pctFewerExamplesTopOne}{61\%}
\newcommand{\pctFewerExamplesTopOneUpperBound}{65\%}
\newcommand{\pctFewerExamplesBaseCaseNoSketch}{58\%}
\newcommand{\pctFewerExamplesBaseCaseStrategy}{45\%}

\newcommand{\benchmarkName}[1]
  {#1}
\newcommand{\benchmarkNameBool}[1]
  {\benchmarkName{bool\_#1}}
\newcommand{\benchmarkNameList}[1]
  {\benchmarkName{list\_#1}}
\newcommand{\benchmarkNameNat}[1]
  {\benchmarkName{nat\_#1}}
\newcommand{\benchmarkNameTree}[1]
  {\benchmarkName{tree\_#1}}

\newcommand{\benchmarkTypeGap}
  {&&&&&&&&&&}

\newcommand{\benchmarkExperimentOne}[3]
  {&{#1}&{#3}}
\newcommand{\benchmarkExperimentOneFailedTimeOut}[1]
  {&{#1}&{\scriptsize{timeout}}}
\newcommand{\benchmarkExperimentOneFailedOverSpecialized}[1]
  {&{#1}&{\scriptsize{overspec}}}
\newcommand{\benchmarkExperimentOneFailedNoSolutions}[1]
  {&{#1}&{\scriptsize{none}}}
\newcommand{\benchmarkExperimentThreeFailedTimeOut}
  {&{\scriptsize{timeout}}}
\newcommand{\benchmarkExperimentThreeFailedOverSpecialized}
  {&{\scriptsize{overspec}}}
\newcommand{\benchmarkExperimentTwo}[9]
  {&{#1}
  }
%% TODO rename: used for experiment three also
\newcommand{\benchmarkExperimentTwoRand}[3]
  {&(#1, #2)$^{#3}$}
\newcommand{\benchmarkExperimentTwoRandFailedHigherOrder}
  {&\blankEntry}
\newcommand{\benchmarkExperimentTwoRandFailedTimeout}
  {&\scriptsize{timeout}}
%% TODO rename
\newcommand{\benchmarkExperimentTwoRandFailedNoNinety}
  {&\scriptsize{failed}}
\newcommand{\benchmarkExperimentThree}[9]
  {&{#1}}

\newcommand{\insideOut}
  %% {*}
  {}
\newcommand{\upperBound}
  {*}
\newcommand{\blankEntry}
  {---}
\newcommand{\benchmarkExperimentTwoBlank}
  {&\blankEntry}
\newcommand{\benchmarkExperimentThreeBlank}
  {&\blankEntry}
\newcommand{\displayPct}[1]
  {\phantom{0 (}{#1}\phantom{)}}
\newcommand{\displayPctUpperBound}[1]
  {\phantom{0 (}{#1}\upperBound}

\newcommand{\leonquidBlank}
  {\textcolor{gray}{---}}
\newcommand{\leonquidCorrectNoPhantom}
  {\textcolor{gray}{\cmark}}
\newcommand{\leonquidCorrect}
  {\textcolor{gray}{\cmark\phantom{$^1$}}}
\newcommand{\leonquidIncorrect}
  {\textcolor{orange}{\xmark$^1$}}
\newcommand{\leonquidError}
  {\textcolor{orange}{\xmark$^2$}}
\newcommand{\leonHigherOrderFunc}
  {\textcolor{orange}{\xmark$^3$}}
\newcommand{\synquidNotTraceComplete}
  {\textcolor{orange}{\xmark$^4$}}
\newcommand{\synquidDatatypeAxiomsNoPhantom}
  {\textcolor{orange}{\textbf{?}}}
\newcommand{\synquidDatatypeAxioms}
  {\textcolor{orange}{\textbf{?\ }}}



\begin{figure}

\experimentTableSize

\begin{tabular}{l|cc|cc|cc||cc|cc}
  \multicolumn{7}{c||}{\textbf{\snsMyth{}}}
& \multicolumn{2}{c}{\textbf{\leon{}}}
& \multicolumn{2}{|c}{\textbf{\synquid{}}}
\\\hline
\multicolumn{1}{r|}{\textbf{Experiment}} &
\multicolumn{2}{c|}{\textbf{1}} &
\textbf{2a} & \textbf{2b} & \textbf{3a} & \textbf{3b}
& \multicolumn{2}{c|}{\textbf{4}} & \multicolumn{2}{c}{\textbf{4}}
\\\hline
\multicolumn{1}{r|}{{Sketch}} &
\multicolumn{4}{c|}{\textit{None}} &
\multicolumn{2}{c||}{\textit{Base Case}}
& \multicolumn{2}{c|}{}
\\\hline
\multicolumn{1}{r|}{{\#Benchmarks}} &
\multicolumn{4}{c|}{\textit{\numBenchmarks{}/\numBenchmarksAll{} \myth{} benchmarks}} &
\multicolumn{2}{c||}{\textit{\numBenchmarksBase{}/\numBenchmarksRecursive{} rec. benchmarks}}
& \multicolumn{2}{c|}{}
\\\hline
\multicolumn{1}{r|}{{Objective}} &
\multicolumn{2}{c|}{\makebox[0.38in]{\textit{Top-1}}} &
\multicolumn{2}{c|}{\makebox[0.38in]{\textit{Top-1}}} &
\multicolumn{2}{c||} {\makebox[0.38in]{\textit{Top-1-R}}}
& \multicolumn{2}{c|}{}
\\\hline
\textbf{Name} &
\textbf{Expert} & \textbf{Time} &
\textbf{Expert} & \textbf{Random} &
\textbf{Expert} & \textbf{Random} &
\textbf{1} & \textbf{2a} &
\textbf{1} & \textbf{2a}
\\
&
& &
& {(50\%, 90\%)} &
& {(50\%, 90\%)} &
& &
\\
%% Auto-generated in paper repo by: generate_tables.py
\benchmarkTypeGap
\\
\benchmarkNameBool{band}
\benchmarkExperimentOne{4}{4}{0.004}
\benchmarkExperimentTwo{3 (75\%)}{4}{4}{0.004}{0}{0}{1}{1}{N}
\benchmarkExperimentTwoRand{4}{4}{}
\benchmarkExperimentThreeBlank
& \blankEntry
& \leonquidCorrect
& \leonquidCorrect
& \leonquidCorrect
& \leonquidCorrect
\\
\benchmarkNameBool{bor}
\benchmarkExperimentOne{4}{4}{0.004}
\benchmarkExperimentTwo{3 (75\%)}{4}{4}{0.004}{0}{0}{1}{1}{N}
\benchmarkExperimentTwoRand{4}{4}{}
\benchmarkExperimentThreeBlank
& \blankEntry
& \leonquidCorrect
& \leonquidCorrect
& \leonquidCorrect
& \leonquidCorrect
\\
\benchmarkNameBool{impl}
\benchmarkExperimentOne{4}{1}{0.005}
\benchmarkExperimentTwo{3 (75\%)}{1}{1}{0.005}{0}{0}{1}{1}{N}
\benchmarkExperimentTwoRand{3}{4}{}
\benchmarkExperimentThreeBlank
& \blankEntry
& \leonquidCorrect
& \leonquidCorrect
& \leonquidCorrect
& \leonquidCorrect
\\
\benchmarkNameBool{neg}
\benchmarkExperimentOne{2}{1}{0.002}
\benchmarkExperimentTwo{2 (100\%)}{1}{1}{0.002}{0}{0}{1}{1}{N}
\benchmarkExperimentTwoRand{2}{2}{}
\benchmarkExperimentThreeBlank
& \blankEntry
& \leonquidCorrect
& \leonquidCorrect
& \leonquidCorrect
& \leonquidCorrect
\\
\benchmarkNameBool{xor}
\benchmarkExperimentOne{4}{28}{0.011}
\benchmarkExperimentTwo{3 (75\%)}{4}{8}{0.009}{0}{0}{1}{1}{N}
\benchmarkExperimentTwoRand{4}{4}{}
\benchmarkExperimentThreeBlank
& \blankEntry
& \leonquidCorrect
& \leonquidIncorrect
& \leonquidCorrect
& \leonquidCorrect
\\
\benchmarkTypeGap
\\
\benchmarkNameList{append}
\benchmarkExperimentOne{6}{1}{0.008}
\benchmarkExperimentTwo{4 (67\%)}{1}{1}{0.010}{1}{1}{0}{0}{R}
\benchmarkExperimentTwoRand{3}{5}{}
\benchmarkExperimentThree{1+1 (33\%)}{1}{23}{0.015}{1}{1}{0}{1}{R}
\benchmarkExperimentTwoRand{1+3}{1+5}{}
& \leonquidCorrect
& \leonquidIncorrect
& \synquidDatatypeAxioms
& \synquidNotTraceComplete
\\
\benchmarkNameList{compress\insideOut}
\benchmarkExperimentOneFailedTimeOut{13}
\benchmarkExperimentTwoBlank
& \blankEntry
\benchmarkExperimentThreeBlank
& \blankEntry
& \leonquidError
& \leonquidBlank
& \synquidDatatypeAxioms
& \leonquidBlank
\\
\benchmarkNameList{concat}
\benchmarkExperimentOne{6}{3}{0.008}
\benchmarkExperimentTwo{3 (50\%)}{3}{3}{0.009}{1}{1}{0}{0}{R}
\benchmarkExperimentTwoRand{2}{3}{}
\benchmarkExperimentThree{1+2 (50\%)}{8}{17}{0.011}{1}{1}{0}{1}{R}
\benchmarkExperimentTwoRand{1+2}{1+3}{}
& \leonquidCorrect
& \leonquidIncorrect
& \synquidDatatypeAxioms
& \synquidNotTraceComplete
\\
\benchmarkNameList{drop}
\benchmarkExperimentOne{11}{30}{0.030}
\benchmarkExperimentTwo{5 (45\%)}{9}{30}{0.017}{1}{1}{0}{0}{R}
\benchmarkExperimentTwoRand{6}{9}{}
\benchmarkExperimentThree{1+2 (27\%)}{2}{10}{0.009}{1}{1}{0}{1}{R}
\benchmarkExperimentTwoRand{1+8}{1+15}{}
& \leonquidCorrect
& \leonquidCorrect
& \synquidDatatypeAxioms
& \synquidNotTraceComplete
\\
\benchmarkNameList{even\_parity}
\benchmarkExperimentOne{7}{29}{0.051}
\benchmarkExperimentTwo{5 (71\%)}{18}{30}{0.016}{1}{1}{0}{0}{R}
\benchmarkExperimentTwoRandFailedNoNinety
\benchmarkExperimentThreeFailedOverSpecialized
\benchmarkExperimentTwoRandFailedNoNinety
& \leonquidCorrect
& \leonquidIncorrect
& \synquidDatatypeAxioms
& \synquidNotTraceComplete
\\
\benchmarkNameList{filter}
\benchmarkExperimentOne{8}{30}{0.130}
\benchmarkExperimentTwo{4 (50\%)}{15}{30}{0.080}{1}{1}{0}{1}{R}
\benchmarkExperimentTwoRandFailedHigherOrder
\benchmarkExperimentThreeFailedOverSpecialized
\benchmarkExperimentTwoRandFailedHigherOrder
& \leonHigherOrderFunc
& \leonHigherOrderFunc
& \leonHigherOrderFunc
& \leonHigherOrderFunc
\\
\benchmarkNameList{fold}
\benchmarkExperimentOne{9}{1}{0.765}
\benchmarkExperimentTwo{3 (33\%)}{1}{30}{0.714}{1}{1}{0}{0}{R}
\benchmarkExperimentTwoRandFailedHigherOrder
\benchmarkExperimentThree{1+3 (44\%)}{9}{30}{2.328}{1}{1}{0}{0}{R}
\benchmarkExperimentTwoRandFailedHigherOrder
& \leonHigherOrderFunc
& \leonHigherOrderFunc
& \leonHigherOrderFunc
& \leonHigherOrderFunc
\\
\benchmarkNameList{hd}
\benchmarkExperimentOne{3}{1}{0.003}
\benchmarkExperimentTwo{2 (67\%)}{1}{3}{0.004}{0}{1}{1}{1}{N}
\benchmarkExperimentTwoRand{2}{3}{}
\benchmarkExperimentThreeBlank
& \blankEntry
& \leonquidCorrect
& \leonquidCorrect
& \synquidDatatypeAxioms
& \leonquidIncorrect
\\
\benchmarkNameList{inc}
\benchmarkExperimentOne{4}{3}{0.184}
\benchmarkExperimentTwo{2 (50\%)}{3}{3}{0.067}{0}{0}{1}{1}{N}
\benchmarkExperimentTwoRand{2}{2}{}
\benchmarkExperimentThreeBlank
& \blankEntry
& \leonquidCorrect
& \leonquidCorrect
& \synquidDatatypeAxioms
& \synquidDatatypeAxioms
\\
\benchmarkNameList{last}
\benchmarkExperimentOne{6}{1}{0.007}
\benchmarkExperimentTwo{4 (67\%)}{1}{1}{0.007}{1}{1}{0}{0}{R}
\benchmarkExperimentTwoRand{5}{12}{}
\benchmarkExperimentThree{1+2 (50\%)}{1}{7}{0.006}{1}{1}{0}{1}{R}
\benchmarkExperimentTwoRand{1+5}{1+8}{}
& \leonquidCorrect
& \leonquidCorrect
& \synquidDatatypeAxioms
& \synquidNotTraceComplete
\\
\benchmarkNameList{length}
\benchmarkExperimentOne{3}{1}{0.003}
\benchmarkExperimentTwo{3 (100\%)}{1}{1}{0.004}{1}{1}{0}{0}{R}
\benchmarkExperimentTwoRand{2}{3}{}
\benchmarkExperimentThree{1+1 (67\%)}{1}{3}{0.004}{1}{1}{0}{1}{R}
\benchmarkExperimentTwoRand{1+2}{1+3}{}
& \leonquidCorrect
& \leonquidBlank
& \synquidDatatypeAxioms
& \leonquidBlank
\\
\benchmarkNameList{map}
\benchmarkExperimentOne{8}{2}{0.039}
\benchmarkExperimentTwo{4 (50\%)}{2}{6}{0.043}{1}{1}{0}{0}{R}
\benchmarkExperimentTwoRandFailedHigherOrder
\benchmarkExperimentThree{1+2 (38\%)}{1}{9}{0.710}{1}{1}{0}{1}{R}
\benchmarkExperimentTwoRandFailedHigherOrder
& \leonHigherOrderFunc
& \leonHigherOrderFunc
& \leonHigherOrderFunc
& \leonHigherOrderFunc
\\
\benchmarkNameList{nth}
\benchmarkExperimentOne{13}{10}{0.113}
\benchmarkExperimentTwo{5 (38\%)}{2}{4}{0.036}{1}{1}{0}{0}{R}
\benchmarkExperimentTwoRand{8}{15}{}
\benchmarkExperimentThree{1+2 (23\%)}{2}{11}{0.008}{1}{1}{0}{1}{R}
\benchmarkExperimentTwoRand{1+8}{1+16}{}
& \leonquidCorrect
& \leonquidCorrect
& \synquidDatatypeAxioms
& \synquidNotTraceComplete
\\
\benchmarkNameList{pairwise\_swap}
\benchmarkExperimentOne{7}{30}{4.229}
\benchmarkExperimentTwo{5 (71\%)}{30}{30}{1.401}{1}{1}{0}{0}{R}
\benchmarkExperimentTwoRandFailedTimeout
\benchmarkExperimentThreeFailedOverSpecialized
\benchmarkExperimentTwoRandFailedTimeout
& \leonquidCorrect
& \leonquidCorrect
& \synquidDatatypeAxioms
& \synquidNotTraceComplete
\\
\benchmarkNameList{rev\_append}
\benchmarkExperimentOne{5}{1}{0.097}
\benchmarkExperimentTwo{3 (60\%)}{1}{1}{0.068}{1}{1}{0}{0}{R}
\benchmarkExperimentTwoRand{5}{9}{}
\benchmarkExperimentThree{1+2 (60\%)}{1}{1}{0.047}{1}{1}{0}{0}{R}
\benchmarkExperimentTwoRand{1+3}{1+14}{}
& \leonquidCorrect
& \leonquidCorrect
& \synquidDatatypeAxioms
& \synquidNotTraceComplete
\\
\benchmarkNameList{rev\_fold}
\benchmarkExperimentOne{5}{2}{0.027}
\benchmarkExperimentTwo{2 (40\%)}{2}{2}{0.025}{0}{0}{1}{1}{N}
\benchmarkExperimentTwoRand{2}{3}{}
\benchmarkExperimentThreeBlank
& \blankEntry
& \leonquidCorrect
& \leonquidCorrect
& \synquidDatatypeAxioms
& \synquidDatatypeAxioms
\\
\benchmarkNameList{rev\_snoc}
\benchmarkExperimentOne{5}{1}{0.009}
\benchmarkExperimentTwo{3 (60\%)}{1}{1}{0.017}{1}{1}{0}{0}{R}
\benchmarkExperimentTwoRand{3}{7}{}
\benchmarkExperimentThree{1+1 (40\%)}{1}{9}{0.050}{1}{1}{0}{1}{R}
\benchmarkExperimentTwoRand{1+3}{1+5}{}
& \leonquidCorrect
& \leonquidCorrect
& \synquidDatatypeAxioms
& \synquidNotTraceComplete
\\
\benchmarkNameList{rev\_tailcall}
\benchmarkExperimentOne{8}{1}{0.007}
\benchmarkExperimentTwo{3 (38\%)}{1}{1}{0.007}{1}{1}{0}{0}{R}
\benchmarkExperimentTwoRand{3}{6}{}
\benchmarkExperimentThree{1+1 (25\%)}{1}{11}{0.015}{1}{1}{0}{1}{R}
\benchmarkExperimentTwoRand{1+3}{1+6}{}
& \leonquidIncorrect
& \leonquidCorrect
& \synquidDatatypeAxioms
& \synquidNotTraceComplete
\\
\benchmarkNameList{snoc}
\benchmarkExperimentOne{8}{6}{0.012}
\benchmarkExperimentTwo{4 (50\%)}{6}{30}{0.013}{1}{1}{0}{1}{R}
\benchmarkExperimentTwoRand{3}{4}{}
\benchmarkExperimentThree{1+2 (38\%)}{1}{6}{0.007}{1}{1}{0}{1}{R}
\benchmarkExperimentTwoRand{1+2}{1+4}{}
& \leonquidCorrect
& \leonquidCorrect
& \synquidDatatypeAxioms
& \synquidNotTraceComplete
\\
\benchmarkNameList{sort\_sorted\_insert}
\benchmarkExperimentOne{7}{1}{0.015}
\benchmarkExperimentTwo{3 (43\%)}{1}{1}{0.015}{1}{1}{0}{0}{R}
\benchmarkExperimentTwoRand{3}{6}{}
\benchmarkExperimentThree{1+1 (29\%)}{2}{24}{0.047}{1}{1}{0}{1}{R}
\benchmarkExperimentTwoRand{1+3}{1+6}{}
& \leonquidCorrect
& \leonquidCorrect
& \synquidDatatypeAxioms
& \synquidNotTraceComplete
\\
\benchmarkNameList{sorted\_insert}
\benchmarkExperimentOne{12}{30}{10.964}
\benchmarkExperimentTwo{7 (58\%)}{20}{30}{2.817}{1}{1}{0}{0}{R}
\benchmarkExperimentTwoRandFailedTimeout
\benchmarkExperimentThreeFailedOverSpecialized
\benchmarkExperimentTwoRandFailedTimeout
& \leonquidError
& \leonquidError
& \synquidDatatypeAxioms
& \synquidNotTraceComplete
\\
\benchmarkNameList{stutter}
\benchmarkExperimentOne{3}{2}{0.004}
\benchmarkExperimentTwo{2 (67\%)}{2}{18}{0.006}{1}{1}{0}{1}{R}
\benchmarkExperimentTwoRand{3}{3}{}
\benchmarkExperimentThree{1+1 (67\%)}{1}{9}{0.005}{1}{1}{0}{1}{R}
\benchmarkExperimentTwoRand{1+3}{1+4}{}
& \leonquidCorrect
& \leonquidCorrect
& \synquidDatatypeAxioms
& \synquidNotTraceComplete
\\
\benchmarkNameList{sum}
\benchmarkExperimentOne{3}{4}{0.023}
\benchmarkExperimentTwo{2 (67\%)}{4}{5}{0.025}{0}{0}{1}{1}{N}
\benchmarkExperimentTwoRand{2}{3}{}
\benchmarkExperimentThreeBlank
& \blankEntry
& \leonquidCorrect
& \leonquidIncorrect
& \synquidDatatypeAxioms
& \synquidDatatypeAxioms
\\
\benchmarkNameList{take}
\benchmarkExperimentOne{12}{30}{0.075}
\benchmarkExperimentTwo{6 (50\%)}{2}{6}{0.045}{1}{1}{0}{0}{R}
\benchmarkExperimentTwoRand{7}{10}{}
\benchmarkExperimentThree{1+3 (33\%)}{5}{7}{0.013}{1}{1}{0}{0}{R}
\benchmarkExperimentTwoRand{1+8}{1+15}{}
& \leonquidCorrect
& \leonquidCorrect
& \synquidDatatypeAxioms
& \synquidNotTraceComplete
\\
\benchmarkNameList{tl}
\benchmarkExperimentOne{3}{2}{0.003}
\benchmarkExperimentTwo{2 (67\%)}{2}{6}{0.004}{0}{0}{1}{1}{N}
\benchmarkExperimentTwoRand{2}{3}{}
\benchmarkExperimentThreeBlank
& \blankEntry
& \leonquidCorrect
& \leonquidCorrect
& \leonquidIncorrect
& \leonquidIncorrect
\\
\benchmarkTypeGap
\\
\benchmarkNameNat{add}
\benchmarkExperimentOne{9}{2}{0.007}
\benchmarkExperimentTwo{4 (44\%)}{2}{2}{0.007}{1}{1}{0}{0}{R}
\benchmarkExperimentTwoRand{4}{6}{}
\benchmarkExperimentThree{1+1 (22\%)}{2}{30}{0.008}{1}{1}{0}{1}{R}
\benchmarkExperimentTwoRand{1+4}{1+6}{}
& \leonquidCorrect
& \leonquidCorrect
& \synquidDatatypeAxioms
& \synquidNotTraceComplete
\\
\benchmarkNameNat{iseven}
\benchmarkExperimentOne{4}{1}{0.004}
\benchmarkExperimentTwo{3 (75\%)}{1}{2}{0.005}{1}{1}{0}{1}{R}
\benchmarkExperimentTwoRand{3}{4}{}
\benchmarkExperimentThree{1+2 (75\%)}{1}{2}{0.004}{1}{1}{0}{1}{R}
\benchmarkExperimentTwoRand{1+3}{1+4}{}
& \leonquidCorrect
& \leonquidCorrect
& \synquidDatatypeAxioms
& \synquidNotTraceComplete
\\
\benchmarkNameNat{max}
\benchmarkExperimentOne{9}{30}{0.039}
\benchmarkExperimentTwo{9 (100\%)}{30}{30}{0.039}{1}{1}{0}{0}{R}
\benchmarkExperimentTwoRand{9}{11}{}
\benchmarkExperimentThree{1+4 (56\%)}{13}{20}{0.032}{1}{1}{0}{0}{R}
\benchmarkExperimentTwoRand{1+8}{1+12}{}
& \leonquidIncorrect
& \leonquidBlank
& \synquidDatatypeAxioms
& \leonquidBlank
\\
\benchmarkNameNat{pred}
\benchmarkExperimentOne{3}{2}{0.002}
\benchmarkExperimentTwo{2 (67\%)}{2}{4}{0.003}{0}{0}{1}{1}{N}
\benchmarkExperimentTwoRand{2}{3}{}
\benchmarkExperimentThreeBlank
& \blankEntry
& \leonquidCorrect
& \leonquidCorrect
& \leonquidIncorrect
& \leonquidIncorrect
\\
\benchmarkTypeGap
\\
\benchmarkNameTree{binsert}
\benchmarkExperimentOneFailedTimeOut{20}
\benchmarkExperimentTwoBlank
& \blankEntry
\benchmarkExperimentThreeBlank
& \blankEntry
& \leonquidError
& \leonquidBlank
& \synquidDatatypeAxioms
& \leonquidBlank
\\
\benchmarkNameTree{collect\_leaves}
\benchmarkExperimentOne{6}{9}{0.066}
\benchmarkExperimentTwo{3 (50\%)}{9}{9}{0.044}{1}{1}{0}{0}{R}
\benchmarkExperimentTwoRand{3}{4}{3}
\benchmarkExperimentThree{1+2 (50\%)}{1}{2}{0.022}{1}{1}{0}{1}{R}
\benchmarkExperimentTwoRand{1+3}{1+3}{}
& \leonquidCorrect
& \leonquidCorrect
& \synquidDatatypeAxioms
& \synquidNotTraceComplete
\\
\benchmarkNameTree{count\_leaves}
\benchmarkExperimentOne{7}{30}{3.009}
\benchmarkExperimentTwo{3 (43\%)}{30}{30}{1.152}{1}{1}{0}{0}{R}
\benchmarkExperimentTwoRandFailedTimeout
\benchmarkExperimentThree{1+1 (29\%)}{2}{3}{0.076}{1}{1}{0}{1}{R}
\benchmarkExperimentTwoRandFailedTimeout
& \leonquidCorrect
& \leonquidCorrect
& \synquidDatatypeAxioms
& \synquidNotTraceComplete
\\
\benchmarkNameTree{count\_nodes}
\benchmarkExperimentOne{6}{30}{0.323}
\benchmarkExperimentTwo{3 (50\%)}{30}{30}{0.188}{1}{1}{0}{0}{R}
\benchmarkExperimentTwoRand{4}{$\downarrow$}{10}
\benchmarkExperimentThree{1+2 (50\%)}{18}{18}{0.075}{1}{1}{0}{0}{R}
\benchmarkExperimentTwoRand{1+4}{1+5}{10}
& \leonquidCorrect
& \leonquidCorrect
& \synquidDatatypeAxioms
& \synquidNotTraceComplete
\\
\benchmarkNameTree{inorder}
\benchmarkExperimentOne{5}{9}{0.114}
\benchmarkExperimentTwo{4 (80\%)}{9}{9}{0.100}{1}{1}{0}{0}{R}
\benchmarkExperimentTwoRand{3}{4}{}
\benchmarkExperimentThree{1+2 (60\%)}{1}{2}{0.022}{1}{1}{0}{1}{R}
\benchmarkExperimentTwoRand{1+3}{1+3}{}
& \leonquidCorrect
& \leonquidCorrect
& \synquidDatatypeAxioms
& \synquidNotTraceComplete
\\
\benchmarkNameTree{map}
\benchmarkExperimentOne{7}{12}{0.055}
\benchmarkExperimentTwo{4 (57\%)}{12}{12}{0.056}{1}{1}{0}{0}{R}
\benchmarkExperimentTwoRandFailedHigherOrder
\benchmarkExperimentThree{1+3 (57\%)}{30}{30}{0.707}{1}{1}{0}{0}{R}
\benchmarkExperimentTwoRandFailedHigherOrder
& \leonHigherOrderFunc
& \leonHigherOrderFunc
& \leonHigherOrderFunc
& \leonHigherOrderFunc
\\
\benchmarkNameTree{nodes\_at\_level}
\benchmarkExperimentOneFailedTimeOut{11}
\benchmarkExperimentTwoBlank
& \blankEntry
\benchmarkExperimentThreeBlank
& \blankEntry
& \leonquidError
& \leonquidBlank
& \synquidDatatypeAxioms
& \leonquidBlank
\\
\benchmarkNameTree{postorder\insideOut}
\benchmarkExperimentOneFailedTimeOut{20}
\benchmarkExperimentTwoBlank
& \blankEntry
\benchmarkExperimentThreeBlank
& \blankEntry
& \leonquidCorrect
& \leonquidBlank
& \synquidDatatypeAxioms
& \leonquidBlank
\\
\benchmarkNameTree{preorder}
\benchmarkExperimentOne{5}{30}{0.145}
\benchmarkExperimentTwo{3 (60\%)}{30}{30}{0.137}{1}{1}{0}{0}{R}
\benchmarkExperimentTwoRand{3}{3}{3}
\benchmarkExperimentThree{1+2 (60\%)}{4}{4}{0.026}{1}{1}{0}{0}{R}
\benchmarkExperimentTwoRand{1+3}{1+3}{}
& \leonquidCorrect
& \leonquidCorrect
& \synquidDatatypeAxioms
& \synquidNotTraceComplete
\\
\benchmarkTypeGap
%
\\\hline
\textbf{Averages} &
&
&
\displayPctUpperBound{\pctFewerExamplesTopOne} &
&
\displayPct{\phantom{1+}\pctFewerExamplesBaseCaseStrategy} &
&&
\end{tabular}

\vsepBeforeCaption
  %% \captionsetup{justification=centering}
  \caption{
    Experiments.
      \textbf{Top-1(-R)}:
      1st (recursive) solution valid.
    \textbf{Time}:
       Average of 10 runs, in seconds.
    \\
    \textbf{Averages}:
      Non-blank, non-error rows.
    \textit{*Upper bound: \pctFewerExamplesTopOneUpperBound{} for all
     \numBenchmarksAll{} rows.}
  }
\label{fig:experiments}
\end{figure}

\setcounter{figure}{0}

\begin{table}

\experimentTableSize

\begin{tabular}{l|cccccc}
& \multicolumn{6}{c}{\textbf{Experiment 1}} \\\hline
\textbf{Name} &
\textbf{Expert} & \textbf{Expert} & \textbf{Expert} &
\textbf{Time} & \textbf{Time} & \textbf{Time} \\
&
\textit{Submission:} & \multicolumn{2}{c}{\textit{Revised Artifact:}} &
\textit{Submission:} & \multicolumn{2}{c}{\textit{Revised Artifact:}} \\
&
\textit{Fig. 10} & \textit{Ours} & \textit{Yours} &
\textit{Fig. 10} & \textit{Ours} & \textit{Yours} \\
\\ 
bool\_band&4&4&\highlightRed{$\bullet$}&0.004&0.004&\highlightRed{$\bullet$}\\ 
bool\_bor&4&4&\highlightRed{$\bullet$}&0.004&0.004&\highlightRed{$\bullet$}\\ 
bool\_impl&4&4&\highlightRed{$\bullet$}&0.005&\highlightBlue{0.004}&\highlightRed{$\bullet$}\\ 
bool\_neg&2&2&\highlightRed{$\bullet$}&0.002&\highlightBlue{0.001}&\highlightRed{$\bullet$}\\ 
bool\_xor&4&4&\highlightRed{$\bullet$}&0.011&\highlightBlue{0.009}&\highlightRed{$\bullet$}\\ 
\\ 
list\_append&6&6&\highlightRed{$\bullet$}&0.008&0.008&\highlightRed{$\bullet$}\\ 
list\_compress&13&\highlightBlue{$\bullet$}&$\bullet$&timeout&\highlightBlue{$\bullet$}&$\bullet$\\ 
list\_concat&6&6&\highlightRed{$\bullet$}&0.008&\highlightBlue{0.009}&\highlightRed{$\bullet$}\\ 
list\_drop&11&11&\highlightRed{$\bullet$}&0.030&\highlightBlue{0.093}&\highlightRed{$\bullet$}\\ 
list\_even\_parity&7&\highlightBlue{$\bullet$}&$\bullet$&0.051&\highlightBlue{$\bullet$}&$\bullet$\\ 
list\_filter&8&8&\highlightRed{$\bullet$}&0.130&\highlightBlue{0.125}&\highlightRed{$\bullet$}\\ 
list\_fold&9&9&\highlightRed{$\bullet$}&0.765&\highlightBlue{0.852}&\highlightRed{$\bullet$}\\ 
list\_hd&3&3&\highlightRed{$\bullet$}&0.003&0.003&\highlightRed{$\bullet$}\\ 
list\_inc&4&4&\highlightRed{$\bullet$}&0.184&\highlightBlue{0.018}&\highlightRed{$\bullet$}\\ 
list\_last&6&6&\highlightRed{$\bullet$}&0.007&\highlightBlue{0.008}&\highlightRed{$\bullet$}\\ 
list\_length&3&3&\highlightRed{$\bullet$}&0.003&\highlightBlue{0.002}&\highlightRed{$\bullet$}\\ 
list\_map&8&8&\highlightRed{$\bullet$}&0.039&\highlightBlue{0.048}&\highlightRed{$\bullet$}\\ 
list\_nth&13&13&\highlightRed{$\bullet$}&0.113&\highlightBlue{0.131}&\highlightRed{$\bullet$}\\ 
list\_pairwise\_swap&7&7&\highlightRed{$\bullet$}&4.229&\highlightBlue{1.386}&\highlightRed{$\bullet$}\\ 
list\_rev\_append&5&5&\highlightRed{$\bullet$}&0.097&\highlightBlue{0.111}&\highlightRed{$\bullet$}\\ 
list\_rev\_fold&5&5&\highlightRed{$\bullet$}&0.027&\highlightBlue{0.035}&\highlightRed{$\bullet$}\\ 
list\_rev\_snoc&5&5&\highlightRed{$\bullet$}&0.009&\highlightBlue{0.010}&\highlightRed{$\bullet$}\\ 
list\_rev\_tailcall&8&8&\highlightRed{$\bullet$}&0.007&\highlightBlue{0.008}&\highlightRed{$\bullet$}\\ 
list\_snoc&8&8&\highlightRed{$\bullet$}&0.012&\highlightBlue{0.014}&\highlightRed{$\bullet$}\\ 
list\_sort\_sorted\_insert&7&7&\highlightRed{$\bullet$}&0.015&0.015&\highlightRed{$\bullet$}\\ 
list\_sorted\_insert&12&12&\highlightRed{$\bullet$}&10.964&\highlightBlue{3.034}&\highlightRed{$\bullet$}\\ 
list\_stutter&3&3&\highlightRed{$\bullet$}&0.004&\highlightBlue{0.003}&\highlightRed{$\bullet$}\\ 
list\_sum&3&3&\highlightRed{$\bullet$}&0.023&\highlightBlue{0.029}&\highlightRed{$\bullet$}\\ 
list\_take&12&12&\highlightRed{$\bullet$}&0.075&\highlightBlue{0.070}&\highlightRed{$\bullet$}\\ 
list\_tl&3&3&\highlightRed{$\bullet$}&0.003&\highlightBlue{0.002}&\highlightRed{$\bullet$}\\ 
\\ 
nat\_add&9&9&\highlightRed{$\bullet$}&0.007&\highlightBlue{0.006}&\highlightRed{$\bullet$}\\ 
nat\_iseven&4&4&\highlightRed{$\bullet$}&0.004&\highlightBlue{0.003}&\highlightRed{$\bullet$}\\ 
nat\_max&9&9&\highlightRed{$\bullet$}&0.039&\highlightBlue{0.043}&\highlightRed{$\bullet$}\\ 
nat\_pred&3&3&\highlightRed{$\bullet$}&0.002&\highlightBlue{0.001}&\highlightRed{$\bullet$}\\ 
\\ 
tree\_binsert&20&\highlightBlue{$\bullet$}&$\bullet$&timeout&\highlightBlue{$\bullet$}&$\bullet$\\ 
tree\_collect\_leaves&6&6&\highlightRed{$\bullet$}&0.066&\highlightBlue{0.079}&\highlightRed{$\bullet$}\\ 
tree\_count\_leaves&7&7&\highlightRed{$\bullet$}&3.009&\highlightBlue{2.822}&\highlightRed{$\bullet$}\\ 
tree\_count\_nodes&6&6&\highlightRed{$\bullet$}&0.323&\highlightBlue{0.361}&\highlightRed{$\bullet$}\\ 
tree\_inorder&5&5&\highlightRed{$\bullet$}&0.114&\highlightBlue{0.127}&\highlightRed{$\bullet$}\\ 
tree\_map&7&7&\highlightRed{$\bullet$}&0.055&\highlightBlue{0.065}&\highlightRed{$\bullet$}\\ 
tree\_nodes\_at\_level&11&\highlightBlue{$\bullet$}&$\bullet$&timeout&\highlightBlue{$\bullet$}&$\bullet$\\ 
tree\_postorder&20&\highlightBlue{$\bullet$}&$\bullet$&timeout&\highlightBlue{$\bullet$}&$\bullet$\\ 
tree\_preorder&5&5&\highlightRed{$\bullet$}&0.145&\highlightBlue{0.161}&\highlightRed{$\bullet$}\\ 

\end{tabular}

\vspace{0.10in}

\caption{Experiment 1.}

\end{table}

\begin{table}

\experimentTableSize

\begin{tabular}{l|cccccc}
& \multicolumn{3}{c}{\textbf{Experiment 2a}}
& \multicolumn{3}{c}{\textbf{Experiment 2b}} \\\hline
\textbf{Name} &
\textbf{Expert} & \textbf{Expert} & \textbf{Expert} &
\textbf{Random} & \textbf{Random} & \textbf{Random} \\
&
\textit{Submission:} & \multicolumn{2}{c}{\textit{Revised Artifact:}} &
\textit{Submission:} & \multicolumn{2}{c}{\textit{Revised Artifact:}} \\
&
\textit{Fig. 10} & \textit{Ours} & \textit{Yours} &
\textit{Fig. 10} & \textit{Ours} & \textit{Yours} \\
\\
bool\_band&3 (75\%)&3 (75\%)&\highlightRed{$\bullet$}&(4,4)$^{}$&(4,4)$^{}$&\highlightRed{$\bullet$}\\
bool\_bor&3 (75\%)&3 (75\%)&\highlightRed{$\bullet$}&(4,4)$^{}$&(4,4)$^{}$&\highlightRed{$\bullet$}\\
bool\_impl&3 (75\%)&3 (75\%)&\highlightRed{$\bullet$}&(3,4)$^{}$&\highlightBlue{(4,4)$^{}$}&\highlightRed{$\bullet$}\\
bool\_neg&2 (100\%)&2 (100\%)&\highlightRed{$\bullet$}&(2,2)$^{}$&(2,2)$^{}$&\highlightRed{$\bullet$}\\
bool\_xor&3 (75\%)&\highlightBlue{4 (100\%)}&\highlightRed{$\bullet$}&(4,4)$^{}$&(4,4)$^{}$&\highlightRed{$\bullet$}\\
\\
list\_append&4 (67\%)&4 (67\%)&\highlightRed{$\bullet$}&(3,5)$^{}$&\highlightBlue{(3,4)$^{}$}&\highlightRed{$\bullet$}\\
list\_compress&---&\highlightBlue{$\bullet$}&$\bullet$&---&\highlightBlue{$\bullet$}&$\bullet$\\
list\_concat&3 (50\%)&3 (50\%)&\highlightRed{$\bullet$}&(2,3)$^{}$&\highlightBlue{(3,4)$^{}$}&\highlightRed{$\bullet$}\\
list\_drop&5 (45\%)&5 (45\%)&\highlightRed{$\bullet$}&(6,9)$^{}$&(6,9)$^{}$&\highlightRed{$\bullet$}\\
list\_even\_parity&5 (71\%)&\highlightBlue{$\bullet$}&$\bullet$&\scriptsize{failed}&\highlightBlue{$\bullet$}&$\bullet$\\
list\_filter&4 (50\%)&\highlightBlue{$\bullet$}&$\bullet$&---&\highlightBlue{$\bullet$}&$\bullet$\\
list\_fold&3 (33\%)&3 (33\%)&\highlightRed{$\bullet$}&---&\highlightBlue{$\bullet$}&$\bullet$\\
list\_hd&2 (67\%)&2 (67\%)&\highlightRed{$\bullet$}&(2,3)$^{}$&(2,3)$^{}$&\highlightRed{$\bullet$}\\
list\_inc&2 (50\%)&2 (50\%)&\highlightRed{$\bullet$}&(2,2)$^{}$&(2,2)$^{}$&\highlightRed{$\bullet$}\\
list\_last&4 (67\%)&4 (67\%)&\highlightRed{$\bullet$}&(5,12)$^{}$&\highlightBlue{(5,9)$^{}$}&\highlightRed{$\bullet$}\\
list\_length&3 (100\%)&3 (100\%)&\highlightRed{$\bullet$}&(2,3)$^{}$&\highlightBlue{(3,4)$^{}$}&\highlightRed{$\bullet$}\\
list\_map&4 (50\%)&4 (50\%)&\highlightRed{$\bullet$}&---&\highlightBlue{$\bullet$}&$\bullet$\\
list\_nth&5 (38\%)&5 (38\%)&\highlightRed{$\bullet$}&(8,15)$^{}$&\highlightBlue{(8,12)$^{}$}&\highlightRed{$\bullet$}\\
list\_pairwise\_swap&5 (71\%)&5 (71\%)&\highlightRed{$\bullet$}&\scriptsize{timeout}&\highlightBlue{$\bullet$}&$\bullet$\\
list\_rev\_append&3 (60\%)&3 (60\%)&\highlightRed{$\bullet$}&(5,9)$^{}$&(5,9)$^{}$&\highlightRed{$\bullet$}\\
list\_rev\_fold&2 (40\%)&2 (40\%)&\highlightRed{$\bullet$}&(2,3)$^{}$&\highlightBlue{(3,3)$^{}$}&\highlightRed{$\bullet$}\\
list\_rev\_snoc&3 (60\%)&3 (60\%)&\highlightRed{$\bullet$}&(3,7)$^{}$&\highlightBlue{(3,6)$^{}$}&\highlightRed{$\bullet$}\\
list\_rev\_tailcall&3 (38\%)&3 (38\%)&\highlightRed{$\bullet$}&(3,6)$^{}$&\highlightBlue{(3,5)$^{}$}&\highlightRed{$\bullet$}\\
list\_snoc&4 (50\%)&4 (50\%)&\highlightRed{$\bullet$}&(3,4)$^{}$&(3,4)$^{}$&\highlightRed{$\bullet$}\\
list\_sort\_sorted\_insert&3 (43\%)&3 (43\%)&\highlightRed{$\bullet$}&(3,6)$^{}$&\highlightBlue{(3,5)$^{}$}&\highlightRed{$\bullet$}\\
list\_sorted\_insert&7 (58\%)&7 (58\%)&\highlightRed{$\bullet$}&\scriptsize{timeout}&\highlightBlue{$\bullet$}&$\bullet$\\
list\_stutter&2 (67\%)&2 (67\%)&\highlightRed{$\bullet$}&(3,3)$^{}$&\highlightBlue{(3,4)$^{}$}&\highlightRed{$\bullet$}\\
list\_sum&2 (67\%)&2 (67\%)&\highlightRed{$\bullet$}&(2,3)$^{}$&\highlightBlue{(2,2)$^{}$}&\highlightRed{$\bullet$}\\
list\_take&6 (50\%)&\highlightBlue{5 (42\%)}&\highlightRed{$\bullet$}&(7,10)$^{}$&\highlightBlue{(7,9)$^{}$}&\highlightRed{$\bullet$}\\
list\_tl&2 (67\%)&2 (67\%)&\highlightRed{$\bullet$}&(2,3)$^{}$&(2,3)$^{}$&\highlightRed{$\bullet$}\\
\\
nat\_add&4 (44\%)&4 (44\%)&\highlightRed{$\bullet$}&(4,6)$^{}$&\highlightBlue{(5,6)$^{}$}&\highlightRed{$\bullet$}\\
nat\_iseven&3 (75\%)&3 (75\%)&\highlightRed{$\bullet$}&(3,4)$^{}$&\highlightBlue{(4,4)$^{}$}&\highlightRed{$\bullet$}\\
nat\_max&9 (100\%)&9 (100\%)&\highlightRed{$\bullet$}&(9,11)$^{}$&\highlightBlue{(9,12)$^{}$}&\highlightRed{$\bullet$}\\
nat\_pred&2 (67\%)&2 (67\%)&\highlightRed{$\bullet$}&(2,3)$^{}$&(2,3)$^{}$&\highlightRed{$\bullet$}\\
\\
tree\_binsert&---&\highlightBlue{$\bullet$}&$\bullet$&---&\highlightBlue{$\bullet$}&$\bullet$\\
tree\_collect\_leaves&3 (50\%)&3 (50\%)&\highlightRed{$\bullet$}&(3,4)$^{3}$&\highlightBlue{(3,4)$^{}$}&\highlightRed{$\bullet$}\\
tree\_count\_leaves&3 (43\%)&3 (43\%)&\highlightRed{$\bullet$}&\scriptsize{timeout}&\highlightBlue{$\bullet$}&$\bullet$\\
tree\_count\_nodes&3 (50\%)&3 (50\%)&\highlightRed{$\bullet$}&(4,$\downarrow$)$^{10}$&\highlightBlue{(4,---)$^{}$}&\highlightRed{$\bullet$}\\
tree\_inorder&4 (80\%)&4 (80\%)&\highlightRed{$\bullet$}&(3,4)$^{}$&(3,4)$^{}$&\highlightRed{$\bullet$}\\
tree\_map&4 (57\%)&4 (57\%)&\highlightRed{$\bullet$}&---&\highlightBlue{$\bullet$}&$\bullet$\\
tree\_nodes\_at\_level&---&\highlightBlue{$\bullet$}&$\bullet$&---&\highlightBlue{$\bullet$}&$\bullet$\\
tree\_postorder&---&\highlightBlue{$\bullet$}&$\bullet$&---&\highlightBlue{$\bullet$}&$\bullet$\\
tree\_preorder&3 (60\%)&3 (60\%)&\highlightRed{$\bullet$}&(3,3)$^{3}$&\highlightBlue{(3,3)$^{}$}&\highlightRed{$\bullet$}\\

\end{tabular}

\vspace{0.10in}

\caption{Experiment 2.
%
Differences (in blue) between results from \snsMyth{} at submission and
\snsMyth{} now:
%
\experimentCaptionSize
%
\\[3pt]
%
\textbf{list\_compress, tree\_binsert, tree\_nodes\_at\_level:} Not run because
they failed in Experiment 1.
%
\\[3pt]
%
\textbf{Expert: bool\_xor:} With the algorithmic changes, \snsMyth{} now
requires (all) 4 examples.  (Small changes to search order and search parameters
can change the results of synthesis tools.)
%
\\[3pt]
%
\textbf{Expert: list\_snoc, list\_take:} When looking through our tasks again,
we noticed an opportunity to try removing another example from these benchmarks;
\snsMyth{} produces correct solutions given the fewer examples.
%
\\[3pt]
%
\textbf{Random:} Small variations in k50 and k90 are expected because the
examples are generated randomly. There are some blue dots because our scripts
for benchmarking and generating the table differences do not automatically
display {\scriptsize{failed}}, {\scriptsize{timeout}}, superscripts 3 and 10, or
the $\downarrow$ arrow presented in Figure 10.
%
\\[3pt]
%
}

\end{table}

\begin{table}

\experimentTableSize

\begin{tabular}{l|cccccc}
& \multicolumn{3}{c}{\textbf{Experiment 3a}}
& \multicolumn{3}{c}{\textbf{Experiment 3b}} \\\hline
\textbf{Name} &
\textbf{Expert} & \textbf{Expert} & \textbf{Expert} &
\textbf{Random} & \textbf{Random} & \textbf{Random} \\
&
\textit{Submission:} & \multicolumn{2}{c}{\textit{Revised Artifact:}} &
\textit{Submission:} & \multicolumn{2}{c}{\textit{Revised Artifact:}} \\
&
\textit{Fig. 10} & \textit{Ours} & \textit{Yours} &
\textit{Fig. 10} & \textit{Ours} & \textit{Yours} \\
\\ 
bool\_band&---&\highlightBlue{$\bullet$}&$\bullet$&---&\highlightBlue{$\bullet$}&$\bullet$\\ 
bool\_bor&---&\highlightBlue{$\bullet$}&$\bullet$&---&\highlightBlue{$\bullet$}&$\bullet$\\ 
bool\_impl&---&\highlightBlue{$\bullet$}&$\bullet$&---&\highlightBlue{$\bullet$}&$\bullet$\\ 
bool\_neg&---&\highlightBlue{$\bullet$}&$\bullet$&---&\highlightBlue{$\bullet$}&$\bullet$\\ 
bool\_xor&---&\highlightBlue{$\bullet$}&$\bullet$&---&\highlightBlue{$\bullet$}&$\bullet$\\ 
\\ 
list\_append&1+1 (33\%)&1+1 (33\%)&\highlightRed{$\bullet$}&(1+3,1+5)$^{}$&\highlightBlue{(1+3,1+4)$^{}$}&\highlightRed{$\bullet$}\\ 
list\_compress&---&\highlightBlue{$\bullet$}&$\bullet$&---&\highlightBlue{$\bullet$}&$\bullet$\\ 
list\_concat&1+2 (50\%)&\highlightBlue{1+1 (33\%)}&\highlightRed{$\bullet$}&(1+2,1+3)$^{}$&(1+2,1+3)$^{}$&\highlightRed{$\bullet$}\\ 
list\_drop&1+2 (27\%)&1+2 (27\%)&\highlightRed{$\bullet$}&(1+8,1+15)$^{}$&\highlightBlue{(1+7,1+16)$^{}$}&\highlightRed{$\bullet$}\\ 
list\_even\_parity&\scriptsize{overspec}&\highlightBlue{$\bullet$}&$\bullet$&\scriptsize{failed}&\highlightBlue{$\bullet$}&$\bullet$\\ 
list\_filter&\scriptsize{overspec}&\highlightBlue{1+3 (50\%)}&\highlightRed{$\bullet$}&---&\highlightBlue{$\bullet$}&$\bullet$\\ 
list\_fold&1+3 (44\%)&1+3 (44\%)&\highlightRed{$\bullet$}&---&\highlightBlue{$\bullet$}&$\bullet$\\ 
list\_hd&---&\highlightBlue{$\bullet$}&$\bullet$&---&\highlightBlue{$\bullet$}&$\bullet$\\ 
list\_inc&---&\highlightBlue{$\bullet$}&$\bullet$&---&\highlightBlue{$\bullet$}&$\bullet$\\ 
list\_last&1+2 (50\%)&1+2 (50\%)&\highlightRed{$\bullet$}&(1+5,1+8)$^{}$&\highlightBlue{(1+4,1+8)$^{}$}&\highlightRed{$\bullet$}\\ 
list\_length&1+1 (67\%)&1+1 (67\%)&\highlightRed{$\bullet$}&(1+2,1+3)$^{}$&\highlightBlue{(1+2,1+2)$^{}$}&\highlightRed{$\bullet$}\\ 
list\_map&1+2 (38\%)&1+2 (38\%)&\highlightRed{$\bullet$}&---&\highlightBlue{$\bullet$}&$\bullet$\\ 
list\_nth&1+2 (23\%)&1+2 (23\%)&\highlightRed{$\bullet$}&(1+8,1+16)$^{}$&\highlightBlue{(1+6,1+14)$^{}$}&\highlightRed{$\bullet$}\\ 
list\_pairwise\_swap&\scriptsize{overspec}&\highlightBlue{$\bullet$}&$\bullet$&\scriptsize{timeout}&\highlightBlue{$\bullet$}&$\bullet$\\ 
list\_rev\_append&1+2 (60\%)&1+2 (60\%)&\highlightRed{$\bullet$}&(1+3,1+14)$^{}$&\highlightBlue{(1+3,1+4)$^{}$}&\highlightRed{$\bullet$}\\ 
list\_rev\_fold&---&\highlightBlue{$\bullet$}&$\bullet$&---&\highlightBlue{$\bullet$}&$\bullet$\\ 
list\_rev\_snoc&1+1 (40\%)&1+1 (40\%)&\highlightRed{$\bullet$}&(1+3,1+5)$^{}$&\highlightBlue{(1+3,1+4)$^{}$}&\highlightRed{$\bullet$}\\ 
list\_rev\_tailcall&1+1 (25\%)&1+1 (25\%)&\highlightRed{$\bullet$}&(1+3,1+6)$^{}$&\highlightBlue{(1+3,1+4)$^{}$}&\highlightRed{$\bullet$}\\ 
list\_snoc&1+2 (38\%)&\highlightBlue{1+1 (25\%)}&\highlightRed{$\bullet$}&(1+2,1+4)$^{}$&(1+2,1+4)$^{}$&\highlightRed{$\bullet$}\\ 
list\_sort\_sorted\_insert&1+1 (29\%)&1+1 (29\%)&\highlightRed{$\bullet$}&(1+3,1+6)$^{}$&\highlightBlue{(1+2,1+4)$^{}$}&\highlightRed{$\bullet$}\\ 
list\_sorted\_insert&\scriptsize{overspec}&\highlightBlue{1+7 (67\%)}&\highlightRed{$\bullet$}&\scriptsize{timeout}&\highlightBlue{$\bullet$}&$\bullet$\\ 
list\_stutter&1+1 (67\%)&1+1 (67\%)&\highlightRed{$\bullet$}&(1+3,1+4)$^{}$&\highlightBlue{(1+2,1+3)$^{}$}&\highlightRed{$\bullet$}\\ 
list\_sum&---&\highlightBlue{$\bullet$}&$\bullet$&---&\highlightBlue{$\bullet$}&$\bullet$\\ 
list\_take&1+3 (33\%)&1+3 (33\%)&\highlightRed{$\bullet$}&(1+8,1+15)$^{}$&\highlightBlue{(1+7,1+13)$^{}$}&\highlightRed{$\bullet$}\\ 
list\_tl&---&\highlightBlue{$\bullet$}&$\bullet$&---&\highlightBlue{$\bullet$}&$\bullet$\\ 
\\ 
nat\_add&1+1 (22\%)&1+1 (22\%)&\highlightRed{$\bullet$}&(1+4,1+6)$^{}$&\highlightBlue{(1+3,1+5)$^{}$}&\highlightRed{$\bullet$}\\ 
nat\_iseven&1+2 (75\%)&1+2 (75\%)&\highlightRed{$\bullet$}&(1+3,1+4)$^{}$&(1+3,1+4)$^{}$&\highlightRed{$\bullet$}\\ 
nat\_max&1+4 (56\%)&1+4 (56\%)&\highlightRed{$\bullet$}&(1+8,1+12)$^{}$&\highlightBlue{(1+9,1+12)$^{}$}&\highlightRed{$\bullet$}\\ 
nat\_pred&---&\highlightBlue{$\bullet$}&$\bullet$&---&\highlightBlue{$\bullet$}&$\bullet$\\ 
\\ 
tree\_binsert&---&\highlightBlue{$\bullet$}&$\bullet$&---&\highlightBlue{$\bullet$}&$\bullet$\\ 
tree\_collect\_leaves&1+2 (50\%)&1+2 (50\%)&\highlightRed{$\bullet$}&(1+3,1+3)$^{}$&(1+3,1+3)$^{}$&\highlightRed{$\bullet$}\\ 
tree\_count\_leaves&1+1 (29\%)&1+1 (29\%)&\highlightRed{$\bullet$}&\scriptsize{timeout}&\highlightBlue{$\bullet$}&$\bullet$\\ 
tree\_count\_nodes&1+2 (50\%)&1+2 (50\%)&\highlightRed{$\bullet$}&(1+4,1+5)$^{10}$&\highlightBlue{(1+3,1+4)$^{}$}&\highlightRed{$\bullet$}\\ 
tree\_inorder&1+2 (60\%)&1+2 (60\%)&\highlightRed{$\bullet$}&(1+3,1+3)$^{}$&(1+3,1+3)$^{}$&\highlightRed{$\bullet$}\\ 
tree\_map&1+3 (57\%)&1+3 (57\%)&\highlightRed{$\bullet$}&---&\highlightBlue{$\bullet$}&$\bullet$\\ 
tree\_nodes\_at\_level&---&\highlightBlue{$\bullet$}&$\bullet$&---&\highlightBlue{$\bullet$}&$\bullet$\\ 
tree\_postorder&---&\highlightBlue{$\bullet$}&$\bullet$&---&\highlightBlue{$\bullet$}&$\bullet$\\ 
tree\_preorder&1+2 (60\%)&1+2 (60\%)&\highlightRed{$\bullet$}&(1+3,1+3)$^{}$&(1+3,1+3)$^{}$&\highlightRed{$\bullet$}\\ 

\end{tabular}

\vspace{0.10in}

\caption{Experiment 3.}

\end{table}

\begin{table}

\experimentTableSize

\begin{tabular}{l|cccc}
& \multicolumn{2}{c}{\textbf{\leon{}}}
& \multicolumn{2}{c}{\textbf{\synquid{}}} \\\hline
& \multicolumn{2}{c}{\textbf{4}}
& \multicolumn{2}{c}{\textbf{4}} \\\hline
\textbf{Name} &
\textbf{1} & \textbf{2a} &
\textbf{1} & \textbf{2a} \\
\\
bool\_band&\leonquidCorrect&\leonquidCorrect&\leonquidCorrect&\leonquidCorrect\\
bool\_bor&\leonquidCorrect&\leonquidCorrect&\leonquidCorrect&\leonquidCorrect\\
bool\_impl&\leonquidCorrect&\leonquidCorrect&\leonquidCorrect&\leonquidCorrect\\
bool\_neg&\leonquidCorrect&\leonquidCorrect \highlightBlue{$\Rightarrow$} \leonquidBlank&\leonquidCorrect&\leonquidCorrect \highlightBlue{$\Rightarrow$} \leonquidBlank\\
bool\_xor&\leonquidCorrect&\leonquidIncorrect \highlightBlue{$\Rightarrow$} \leonquidBlank&\leonquidCorrect&\leonquidCorrect \highlightBlue{$\Rightarrow$} \leonquidBlank\\
\\
list\_append&\leonquidCorrect&\leonquidIncorrect&\synquidDatatypeAxioms \highlightBlue{$\Rightarrow$} \leonquidCorrect&\synquidNotTraceComplete \highlightBlue{$\Rightarrow$} \leonquidIncorrect\\
list\_compress&\leonquidError&\leonquidBlank&\synquidDatatypeAxioms \highlightBlue{$\Rightarrow$} \leonquidError&\leonquidBlank\\
list\_concat&\leonquidCorrect&\leonquidIncorrect&\synquidDatatypeAxioms \highlightBlue{$\Rightarrow$} \leonquidIncorrect&\synquidNotTraceComplete \highlightBlue{$\Rightarrow$} \leonquidIncorrect\\
list\_drop&\leonquidCorrect&\leonquidCorrect&\synquidDatatypeAxioms \highlightBlue{$\Rightarrow$} \leonquidCorrect&\synquidNotTraceComplete \highlightBlue{$\Rightarrow$} \leonquidError\\
list\_even\_parity&\leonquidCorrect&\leonquidIncorrect \highlightBlue{$\Rightarrow$} \leonquidBlank&\synquidDatatypeAxioms \highlightBlue{$\Rightarrow$} \leonquidError&\synquidNotTraceComplete \highlightBlue{$\Rightarrow$} \leonquidBlank\\
list\_filter&\leonHigherOrderFunc \highlightBlue{$\Rightarrow$} X&\leonHigherOrderFunc \highlightBlue{$\Rightarrow$} X&\leonHigherOrderFunc \highlightBlue{$\Rightarrow$} X&\leonHigherOrderFunc \highlightBlue{$\Rightarrow$} X\\
list\_fold&\leonHigherOrderFunc \highlightBlue{$\Rightarrow$} X&\leonHigherOrderFunc \highlightBlue{$\Rightarrow$} X&\leonHigherOrderFunc \highlightBlue{$\Rightarrow$} X&\leonHigherOrderFunc \highlightBlue{$\Rightarrow$} X\\
list\_hd&\leonquidCorrect&\leonquidCorrect&\synquidDatatypeAxioms \highlightBlue{$\Rightarrow$} \leonquidCorrect&\leonquidIncorrect \highlightBlue{$\Rightarrow$} \leonquidCorrect\\
list\_inc&\leonquidCorrect&\leonquidCorrect&\synquidDatatypeAxioms \highlightBlue{$\Rightarrow$} \leonquidError&\synquidDatatypeAxioms \highlightBlue{$\Rightarrow$} \leonquidIncorrect\\
list\_last&\leonquidCorrect&\leonquidCorrect&\synquidDatatypeAxioms \highlightBlue{$\Rightarrow$} \leonquidCorrect&\synquidNotTraceComplete \highlightBlue{$\Rightarrow$} \leonquidError\\
list\_length&\leonquidCorrect&\leonquidBlank&\synquidDatatypeAxioms \highlightBlue{$\Rightarrow$} \leonquidCorrect&\leonquidBlank\\
list\_map&\leonHigherOrderFunc \highlightBlue{$\Rightarrow$} X&\leonHigherOrderFunc \highlightBlue{$\Rightarrow$} X&\leonHigherOrderFunc \highlightBlue{$\Rightarrow$} X&\leonHigherOrderFunc \highlightBlue{$\Rightarrow$} X\\
list\_nth&\leonquidCorrect&\leonquidCorrect&\synquidDatatypeAxioms \highlightBlue{$\Rightarrow$} \leonquidCorrect&\synquidNotTraceComplete \highlightBlue{$\Rightarrow$} \leonquidError\\
list\_pairwise\_swap&\leonquidCorrect&\leonquidCorrect&\synquidDatatypeAxioms \highlightBlue{$\Rightarrow$} \leonquidError&\synquidNotTraceComplete \highlightBlue{$\Rightarrow$} \leonquidError\\
list\_rev\_append&\leonquidCorrect&\leonquidCorrect&\synquidDatatypeAxioms \highlightBlue{$\Rightarrow$} \leonquidError&\synquidNotTraceComplete \highlightBlue{$\Rightarrow$} \leonquidError\\
list\_rev\_fold&\leonquidCorrect&\leonquidCorrect&\synquidDatatypeAxioms \highlightBlue{$\Rightarrow$} \leonquidError&\synquidDatatypeAxioms \highlightBlue{$\Rightarrow$} \leonquidError\\
list\_rev\_snoc&\leonquidCorrect&\leonquidCorrect&\synquidDatatypeAxioms \highlightBlue{$\Rightarrow$} \leonquidIncorrect&\synquidNotTraceComplete \highlightBlue{$\Rightarrow$} \leonquidError\\
list\_rev\_tailcall&\leonquidIncorrect&\leonquidCorrect&\synquidDatatypeAxioms \highlightBlue{$\Rightarrow$} \leonquidCorrect&\synquidNotTraceComplete \highlightBlue{$\Rightarrow$} \leonquidIncorrect\\
list\_snoc&\leonquidCorrect&\leonquidCorrect&\synquidDatatypeAxioms \highlightBlue{$\Rightarrow$} \leonquidCorrect&\synquidNotTraceComplete \highlightBlue{$\Rightarrow$} \leonquidError\\
list\_sort\_sorted\_insert&\leonquidCorrect&\leonquidCorrect&\synquidDatatypeAxioms \highlightBlue{$\Rightarrow$} \leonquidError&\synquidNotTraceComplete \highlightBlue{$\Rightarrow$} \leonquidIncorrect\\
list\_sorted\_insert&\leonquidError&\leonquidError&\synquidDatatypeAxioms \highlightBlue{$\Rightarrow$} \leonquidError&\synquidNotTraceComplete \highlightBlue{$\Rightarrow$} \leonquidError\\
list\_stutter&\leonquidCorrect&\leonquidCorrect&\synquidDatatypeAxioms \highlightBlue{$\Rightarrow$} \leonquidCorrect&\synquidNotTraceComplete \highlightBlue{$\Rightarrow$} \leonquidIncorrect\\
list\_sum&\leonquidCorrect&\leonquidIncorrect&\synquidDatatypeAxioms \highlightBlue{$\Rightarrow$} \leonquidError&\synquidDatatypeAxioms \highlightBlue{$\Rightarrow$} \leonquidError\\
list\_take&\leonquidCorrect&\leonquidCorrect&\synquidDatatypeAxioms \highlightBlue{$\Rightarrow$} \leonquidCorrect&\synquidNotTraceComplete \highlightBlue{$\Rightarrow$} \leonquidError\\
list\_tl&\leonquidCorrect&\leonquidCorrect&\leonquidIncorrect \highlightBlue{$\Rightarrow$} \leonquidCorrect&\leonquidIncorrect \highlightBlue{$\Rightarrow$} \leonquidCorrect\\
\\
nat\_add&\leonquidCorrect&\leonquidCorrect&\synquidDatatypeAxioms \highlightBlue{$\Rightarrow$} \leonquidCorrect&\synquidNotTraceComplete \highlightBlue{$\Rightarrow$} \leonquidIncorrect\\
nat\_iseven&\leonquidCorrect&\leonquidCorrect&\synquidDatatypeAxioms \highlightBlue{$\Rightarrow$} \leonquidCorrect&\synquidNotTraceComplete \highlightBlue{$\Rightarrow$} \leonquidError\\
nat\_max&\leonquidIncorrect&\leonquidBlank&\synquidDatatypeAxioms \highlightBlue{$\Rightarrow$} \leonquidCorrect&\leonquidBlank\\
nat\_pred&\leonquidCorrect&\leonquidCorrect&\leonquidIncorrect \highlightBlue{$\Rightarrow$} \leonquidCorrect&\leonquidIncorrect \highlightBlue{$\Rightarrow$} \leonquidCorrect\\
\\
tree\_binsert&\leonquidError&\leonquidBlank&\synquidDatatypeAxioms \highlightBlue{$\Rightarrow$} \leonquidError&\leonquidBlank\\
tree\_collect\_leaves&\leonquidCorrect&\leonquidCorrect&\synquidDatatypeAxioms \highlightBlue{$\Rightarrow$} \leonquidIncorrect&\synquidNotTraceComplete \highlightBlue{$\Rightarrow$} \leonquidIncorrect\\
tree\_count\_leaves&\leonquidCorrect&\leonquidCorrect&\synquidDatatypeAxioms \highlightBlue{$\Rightarrow$} \leonquidError&\synquidNotTraceComplete \highlightBlue{$\Rightarrow$} \leonquidError\\
tree\_count\_nodes&\leonquidCorrect&\leonquidCorrect&\synquidDatatypeAxioms \highlightBlue{$\Rightarrow$} \leonquidIncorrect&\synquidNotTraceComplete \highlightBlue{$\Rightarrow$} \leonquidError\\
tree\_inorder&\leonquidCorrect&\leonquidCorrect&\synquidDatatypeAxioms \highlightBlue{$\Rightarrow$} \leonquidIncorrect&\synquidNotTraceComplete \highlightBlue{$\Rightarrow$} \leonquidError\\
tree\_map&\leonHigherOrderFunc \highlightBlue{$\Rightarrow$} X&\leonHigherOrderFunc \highlightBlue{$\Rightarrow$} X&\leonHigherOrderFunc \highlightBlue{$\Rightarrow$} X&\leonHigherOrderFunc \highlightBlue{$\Rightarrow$} X\\
tree\_nodes\_at\_level&\leonquidError&\leonquidBlank&\synquidDatatypeAxioms \highlightBlue{$\Rightarrow$} \leonquidError&\leonquidBlank\\
tree\_postorder&\leonquidCorrect&\leonquidBlank&\synquidDatatypeAxioms \highlightBlue{$\Rightarrow$} \leonquidError&\leonquidBlank\\
tree\_preorder&\leonquidCorrect&\leonquidCorrect&\synquidDatatypeAxioms \highlightBlue{$\Rightarrow$} \leonquidIncorrect&\synquidNotTraceComplete \highlightBlue{$\Rightarrow$} \leonquidIncorrect\\

\end{tabular}

\vspace{0.10in}

\caption{Experiment 4.
%
Differences (in blue) between results from submission and now:
%
\experimentCaptionSize
%
\\[3pt]
%
blah
%
}

\end{table}


\end{document}
\endinput
