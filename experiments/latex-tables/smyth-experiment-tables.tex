\documentclass[acmsmall,nonacm]{acmart}

\AtBeginDocument{%
  \providecommand\BibTeX{{%
    \normalfont B\kern-0.5em{\scshape i\kern-0.25em b}\kern-0.8em\TeX}}}

\usepackage{pifont}% http://ctan.org/pkg/pifont
\newcommand{\cmark}{\ding{51}}%
\newcommand{\xmark}{\ding{55}}%

\newcommand{\highlightBlue}[1]{\textcolor{blue}{\textbf{#1}}}
\newcommand{\highlightRed}[1]{\textcolor{red}{\textbf{#1}}}

\newcommand{\snsMyth}
  {\ensuremath{\textsc{Smyth}}}
\newcommand{\myth}
  {\ensuremath{\textsc{Myth}}}
\newcommand{\leon}
  {\ensuremath{\textsc{Leon}}}
\newcommand{\synquid}
  {\ensuremath{\textsc{Synquid}}}

\newcommand{\experimentTableSize}
  {\scriptsize}
\newcommand{\experimentCaptionSize}
  {\footnotesize}

\newcommand{\vsepBeforeCaption}{\vspace{0.05in}}

\begin{document}

\title{Smyth Experiment Tables}
%% \subtitle{May 2020}
\maketitle

\vspace{0.30in}
\noindent
\textbf{Figure 10 Results.}
%
The following page replicates Figure 10 from our paper.
%
In Tables 1, 2, and 3:
%
the \textit{Ours} column shows the results obtained and summarized by
the scripts on our machine;
%
the \textit{Fig. 10} columns displays these results, using labels and
colors to explain not-applicable or failed tasks; and
%
the \textit{Yours} column shows the results that you obtained
on your machine; differences from ours are highlighted in red.
%
If you have not run the experiments, the column should be filled with dots.


\vspace{0.30in}
%
\noindent
%
\textbf{\leon{} and \synquid{} Benchmarks (\texttt{experiments/exp-4-logic/}):}
%
For Experiment 4, we wrote a script
%
(\texttt{generate-benchmarks.py})
%
to generate \leon{} and \synquid{} tasks
%
(\texttt{generated/}),
%
which we copied and pasted into the respective web editors.
%
Our labeled results are in \texttt{results/}.
%
The helper script \texttt{show-results.sh} summarizes the labels,
and also includes comments with some shell commands we found useful during
this experiment.
%
The sketching benchmarks we tried
can be found in \texttt{sketches/}.

\clearpage

\setcounter{figure}{9}
\newcommand{\numBenchmarks}{38}
\newcommand{\numBenchmarksAll}{43}

\newcommand{\pctFewerExamplesTopOne}{61\%}
\newcommand{\pctFewerExamplesTopOneUpperBound}{66\%}

\newcommand{\numBenchmarksRecursiveAll}{32}
\newcommand{\numBenchmarksRecursive}{27}
\newcommand{\numBenchmarksBase}{25}

\newcommand{\pctFewerExamplesBaseCaseNoSketch}{57\%}
\newcommand{\pctFewerExamplesBaseCaseStrategy}{46\%}

\newcommand{\maxExamplesBase}{7}
\newcommand{\avgExamplesBase}{2.12}

\newcommand{\benchmarkName}[1]
  {#1}
\newcommand{\benchmarkNameBool}[1]
  {\benchmarkName{bool\_#1}}
\newcommand{\benchmarkNameList}[1]
  {\benchmarkName{list\_#1}}
\newcommand{\benchmarkNameNat}[1]
  {\benchmarkName{nat\_#1}}
\newcommand{\benchmarkNameTree}[1]
  {\benchmarkName{tree\_#1}}

%% \newcommand{\benchmarkTypeGap}
%%   {&&&&&&&&&&}

\definecolor{skippedColor}{HTML}{c0c0c0}

\newcommand{\labelColorSkipped}[1]
  {\textcolor{skippedColor}{#1}}
\newcommand{\labelColorFailed}[1]
  {\textcolor{orange}{#1}}
\newcommand{\labelTimeout}
  {\labelColorFailed{timeout}}
\newcommand{\labelOverspec}
  {\labelColorFailed{overspec}}
\newcommand{\labelRandomFailed}
  %% {\labelColorFailed{failed}}
  {\labelColorFailed{(---,---)}}
\newcommand{\labelIncorrect}
  {\labelColorFailed{incorrect}}
\newcommand{\labelBlank}
  %% {---}
  {$\bullet$}
\newcommand{\labelBlankOneFailed}
  {\labelColorSkipped{\labelBlank$^1$}}
\newcommand{\labelBlankHigherOrder}
  {\labelColorSkipped{\labelBlank$^2$}}
\newcommand{\labelBlankNonRec}
  {\labelColorSkipped{\labelBlank$^3$}}
\newcommand{\labelBlankSameExpertExamples}
  {\labelColorSkipped{\labelBlank$^4$}}
\newcommand{\labelRandomTime}[1]
  %% {\textcolor{orange}{#1}}
  %% {#1}
  {t=#1}

\newcommand{\benchmarkExperimentOne}[3]
  {&{#1}&{#3}}
\newcommand{\benchmarkExperimentOneFailedTimeOut}[1]
  {&{#1}&{\scriptsize{timeout}}}
\newcommand{\benchmarkExperimentOneFailedOverSpecialized}[1]
  {&{#1}&{\scriptsize{overspec}}}
\newcommand{\benchmarkExperimentOneFailedNoSolutions}[1]
  {&{#1}&{\scriptsize{none}}}
\newcommand{\benchmarkExperimentThreeFailedTimeOut}
  {&{\scriptsize{timeout}}}
\newcommand{\benchmarkExperimentThreeFailedOverSpecialized}
  {&{\scriptsize{overspec}}}
\newcommand{\benchmarkExperimentTwo}[9]
  {&{#1}
  }
%% TODO rename: used for experiment three also
\newcommand{\benchmarkExperimentTwoRand}[3]
  {&(#1, #2)$^{#3}$}
\newcommand{\benchmarkExperimentTwoRandFailedHigherOrder}
  {&\blankEntry}
\newcommand{\benchmarkExperimentTwoRandFailedTimeout}
  {&\scriptsize{timeout}}
%% TODO rename
\newcommand{\benchmarkExperimentTwoRandFailedNoNinety}
  {&\scriptsize{failed}}
\newcommand{\benchmarkExperimentThree}[9]
  {&{#1}}

\newcommand{\insideOut}
  %% {*}
  {}
\newcommand{\upperBound}
  {*}
\newcommand{\blankEntry}
  {---}
\newcommand{\benchmarkExperimentTwoBlank}
  {&\blankEntry}
\newcommand{\benchmarkExperimentThreeBlank}
  {&\blankEntry}
\newcommand{\displayPct}[1]
  {\phantom{0 (}{#1}\phantom{)}}
\newcommand{\displayPctUpperBound}[1]
  {\phantom{0 (}{#1}\upperBound}

\newcommand{\leonquidBlank}
  {\textcolor{gray}{---}}
\newcommand{\leonquidCorrectNoPhantom}
  {\textcolor{gray}{\cmark}}
\newcommand{\leonquidCorrect}
  {\textcolor{gray}{\cmark\phantom{$^1$}}}
\newcommand{\leonquidIncorrect}
  {\textcolor{orange}{\xmark$^1$}}
\newcommand{\leonquidError}
  {\textcolor{orange}{\xmark$^2$}}
\newcommand{\leonquidHigherOrderFunc}
  {\textcolor{orange}{\xmark$^3$}}
\newcommand{\synquidNotTraceComplete}
  {\textcolor{orange}{\xmark$^4$}}
\newcommand{\synquidDatatypeAxiomsNoPhantom}
  {\textcolor{orange}{\textbf{?}}}
\newcommand{\synquidDatatypeAxioms}
  {\textcolor{orange}{\textbf{?\ }}}

\begin{figure}

\experimentTableSize

\begin{tabular}{l|cc|cc|cc||cc|cc}
& \multicolumn{6}{c||}{\textbf{\snsMyth{}}}
& \multicolumn{2}{c}{\textbf{\leon{}}}
& \multicolumn{2}{|c}{\textbf{\synquid{}}}
\\\hline
\multicolumn{1}{r|}{\textbf{Experiment}} &
\multicolumn{2}{c|}{\textbf{1}} &
\textbf{2a} & \textbf{2b} & \textbf{3a} & \textbf{3b}
& \multicolumn{2}{c|}{\textbf{4}} & \multicolumn{2}{c}{\textbf{4}}
\\\hline
\multicolumn{1}{r|}{{Sketch / Objective}} &
\multicolumn{2}{c|}{\textit{None / Top-1}} &
\multicolumn{2}{c|}{\textit{None / Top-1}} &
\multicolumn{2}{c||}{\textit{Base Case / Top-1-R}}
%% \multicolumn{1}{r|}{{Sketch}} &
%% \multicolumn{4}{c|}{\textit{None}} &
%% \multicolumn{2}{c||}{\textit{Base Case}}
& \multicolumn{2}{c|}{}
\\\hline
%% \multicolumn{1}{r|}{{\#Benchmarks}} &
%% \multicolumn{4}{c|}{\textit{\numBenchmarks{}/\numBenchmarksAll{} \myth{} benchmarks}} &
%% \multicolumn{2}{c||}{\textit{\numBenchmarksBase{}/\numBenchmarksRecursive{} rec. benchmarks}}
%% \multicolumn{4}{c|}{\textit{\numBenchmarksAll{} \myth{} benchmarks}} &
%% \multicolumn{2}{c||}{\textit{\numBenchmarksRecursiveAll{} recursive benchmarks}}
%% & \multicolumn{2}{c|}{}
%% \\\hline
%% \multicolumn{1}{r|}{{Objective}} &
%% \multicolumn{2}{c|}{\makebox[0.38in]{\textit{Top-1}}} &
%% \multicolumn{2}{c|}{\makebox[0.38in]{\textit{Top-1}}} &
%% \multicolumn{2}{c||} {\makebox[0.38in]{\textit{Top-1-R}}}
%% & \multicolumn{2}{c|}{}
%% \\\hline
\textbf{Name} &
\textbf{Expert} & \textbf{Time} &
\textbf{Expert} & \textbf{Random} &
\textbf{Expert} & \textbf{Random} &
\textbf{1} & \textbf{2a} &
\textbf{1} & \textbf{2a}
\\
&
& &
& {(50\%, 90\%)} &
& {(50\%, 90\%)} &
& &
\\
&&&&&&&&&&\\
bool\_band&4&0.004&3 (75\%)&(4,4)$^{}$&\labelBlankNonRec&\labelBlankNonRec&\leonquidCorrect&\leonquidCorrect&\leonquidCorrect&\leonquidCorrect\\
bool\_bor&4&0.003&3 (75\%)&(4,4)$^{}$&\labelBlankNonRec&\labelBlankNonRec&\leonquidCorrect&\leonquidCorrect&\leonquidCorrect&\leonquidCorrect\\
bool\_impl&4&0.004&3 (75\%)&(4,4)$^{}$&\labelBlankNonRec&\labelBlankNonRec&\leonquidCorrect&\leonquidCorrect&\leonquidCorrect&\leonquidCorrect\\
bool\_neg&2&0.001&2 (100\%)&(2,2)$^{}$&\labelBlankNonRec&\labelBlankNonRec&\leonquidCorrect&\labelBlankSameExpertExamples&\leonquidCorrect&\labelBlankSameExpertExamples\\
bool\_xor&4&0.009&4 (100\%)&(4,4)$^{}$&\labelBlankNonRec&\labelBlankNonRec&\leonquidCorrect&\labelBlankSameExpertExamples&\leonquidCorrect&\labelBlankSameExpertExamples\\
&&&&&&&&&&\\
list\_append&6&0.008&4 (67\%)&(3,4)$^{}$&1+1 (33\%)&(1+3,1+4)$^{}$&\leonquidCorrect&\leonquidIncorrect&\leonquidCorrect&\leonquidIncorrect\\
list\_compress&13&\labelTimeout&\labelBlankOneFailed&\labelBlankOneFailed&\labelBlankOneFailed&\labelBlankOneFailed&\labelBlankOneFailed&\labelBlankOneFailed&\labelBlankOneFailed&\labelBlankOneFailed\\
list\_concat&6&0.010&3 (50\%)&(2,4)$^{}$&\labelIncorrect&(1+3,1+5)$^{}$&\leonquidCorrect&\leonquidIncorrect&\leonquidIncorrect&\leonquidIncorrect\\
list\_drop&11&0.092&5 (45\%)&(6,9)$^{}$&1+2 (27\%)&\labelColorFailed{(1+7,$\downarrow$)$^{}$}&\leonquidCorrect&\leonquidCorrect&\leonquidCorrect&\leonquidError\\
list\_even\_parity&7&\labelOverspec&\labelBlankOneFailed&\labelRandomFailed&\labelBlankOneFailed&\labelRandomFailed&\labelBlankOneFailed&\labelBlankOneFailed&\labelBlankOneFailed&\labelBlankOneFailed\\
list\_filter&\phantom{*}9*&0.144&5 (56\%)&\labelBlankHigherOrder&1+4 (56\%)&\labelBlankHigherOrder&\leonquidHigherOrderFunc&\leonquidHigherOrderFunc&\leonquidHigherOrderFunc&\leonquidHigherOrderFunc\\
list\_fold&9&0.838&3 (33\%)&\labelBlankHigherOrder&1+3 (44\%)&\labelBlankHigherOrder&\leonquidHigherOrderFunc&\leonquidHigherOrderFunc&\leonquidHigherOrderFunc&\leonquidHigherOrderFunc\\
list\_hd&3&0.003&2 (67\%)&(2,3)$^{}$&\labelBlankNonRec&\labelBlankNonRec&\leonquidCorrect&\leonquidCorrect&\leonquidCorrect&\leonquidCorrect\\
list\_inc&4&0.018&2 (50\%)&(2,2)$^{}$&\labelBlankNonRec&\labelBlankNonRec&\leonquidCorrect&\leonquidCorrect&\leonquidError&\leonquidIncorrect\\
list\_last&6&0.007&4 (67\%)&(5,9)$^{}$&1+2 (50\%)&(1+5,1+10)$^{}$&\leonquidCorrect&\leonquidCorrect&\leonquidCorrect&\leonquidError\\
list\_length&3&0.002&3 (100\%)&(3,4)$^{}$&1+1 (67\%)&(1+2,1+2)$^{}$&\leonquidCorrect&\labelBlankSameExpertExamples&\leonquidCorrect&\labelBlankSameExpertExamples\\
list\_map&8&0.049&4 (50\%)&\labelBlankHigherOrder&1+2 (38\%)&\labelBlankHigherOrder&\leonquidHigherOrderFunc&\leonquidHigherOrderFunc&\leonquidHigherOrderFunc&\leonquidHigherOrderFunc\\
list\_nth&13&0.124&5 (38\%)&(7,14)$^{}$&1+2 (23\%)&(1+7,1+15)$^{}$&\leonquidCorrect&\leonquidCorrect&\leonquidCorrect&\leonquidError\\
list\_pairwise\_swap&7&0.634&5 (71\%)&\labelTimeout&\labelOverspec&\labelTimeout&\leonquidCorrect&\leonquidCorrect&\leonquidError&\leonquidError\\
list\_rev\_append&5&0.107&3 (60\%)&(5,8)$^{}$&1+2 (60\%)&(1+3,1+4)$^{}$&\leonquidCorrect&\leonquidCorrect&\leonquidError&\leonquidError\\
list\_rev\_fold&5&0.035&2 (40\%)&(2,4)$^{}$&\labelBlankNonRec&\labelBlankNonRec&\leonquidCorrect&\leonquidCorrect&\leonquidError&\leonquidError\\
list\_rev\_snoc&5&0.010&3 (60\%)&(3,6)$^{}$&1+1 (40\%)&(1+2,1+4)$^{}$&\leonquidCorrect&\leonquidCorrect&\leonquidIncorrect&\leonquidError\\
list\_rev\_tailcall&8&0.008&3 (38\%)&(3,4)$^{}$&1+1 (25\%)&(1+3,1+5)$^{}$&\leonquidIncorrect&\leonquidCorrect&\leonquidCorrect&\leonquidIncorrect\\
list\_snoc&8&0.012&3 (38\%)&(3,4)$^{}$&1+1 (25\%)&(1+3,1+4)$^{}$&\leonquidCorrect&\leonquidCorrect&\leonquidCorrect&\leonquidError\\
list\_sort\_sorted\_insert&7&0.015&3 (43\%)&(3,6)$^{}$&1+1 (29\%)&(1+2,1+4)$^{}$&\leonquidCorrect&\leonquidCorrect&\leonquidError&\leonquidIncorrect\\
list\_sorted\_insert&12&2.902&7 (58\%)&\labelTimeout&1+7 (67\%)&\labelTimeout&\leonquidError&\leonquidError&\leonquidError&\leonquidError\\
list\_stutter&3&0.003&2 (67\%)&(3,3)$^{}$&1+1 (67\%)&(1+2,1+3)$^{}$&\leonquidCorrect&\leonquidCorrect&\leonquidCorrect&\leonquidIncorrect\\
list\_sum&3&0.029&2 (67\%)&(2,2)$^{}$&\labelBlankNonRec&\labelBlankNonRec&\leonquidCorrect&\leonquidIncorrect&\leonquidError&\leonquidError\\
list\_take&12&0.065&5 (42\%)&(6,9)$^{}$&1+3 (33\%)&(1+7,1+16)$^{}$&\leonquidCorrect&\leonquidCorrect&\leonquidCorrect&\leonquidError\\
list\_tl&3&0.002&2 (67\%)&(2,3)$^{}$&\labelBlankNonRec&\labelBlankNonRec&\leonquidCorrect&\leonquidCorrect&\leonquidCorrect&\leonquidCorrect\\
&&&&&&&&&&\\
nat\_add&9&0.006&4 (44\%)&(5,6)$^{}$&1+1 (22\%)&(1+3,1+4)$^{}$&\leonquidCorrect&\leonquidCorrect&\leonquidCorrect&\leonquidIncorrect\\
nat\_iseven&4&0.003&3 (75\%)&(4,4)$^{}$&1+2 (75\%)&(1+3,1+4)$^{}$&\leonquidCorrect&\leonquidCorrect&\leonquidCorrect&\leonquidError\\
nat\_max&9&0.041&9 (100\%)&(8,12)$^{}$&1+4 (56\%)&(1+8,1+12)$^{}$&\leonquidIncorrect&\labelBlankSameExpertExamples&\leonquidCorrect&\labelBlankSameExpertExamples\\
nat\_pred&3&0.001&2 (67\%)&(2,3)$^{}$&\labelBlankNonRec&\labelBlankNonRec&\leonquidCorrect&\leonquidCorrect&\leonquidCorrect&\leonquidCorrect\\
&&&&&&&&&&\\
tree\_binsert&20&\labelTimeout&\labelBlankOneFailed&\labelBlankOneFailed&\labelBlankOneFailed&\labelBlankOneFailed&\labelBlankOneFailed&\labelBlankOneFailed&\labelBlankOneFailed&\labelBlankOneFailed\\
tree\_collect\_leaves&6&0.074&3 (50\%)&(3,4)$^{}$$^{\labelRandomTime{3}}$&1+2 (50\%)&(1+3,1+3)$^{}$&\leonquidCorrect&\leonquidCorrect&\leonquidIncorrect&\leonquidIncorrect\\
tree\_count\_leaves&7&2.660&3 (43\%)&\labelTimeout&1+1 (29\%)&\labelTimeout&\leonquidCorrect&\leonquidCorrect&\leonquidError&\leonquidError\\
tree\_count\_nodes&6&0.351&3 (50\%)&\labelColorFailed{(4,$\downarrow$)$^{}$}$^{\labelRandomTime{10}}$&1+2 (50\%)&(1+3,1+5)$^{}$$^{\labelRandomTime{3}}$&\leonquidCorrect&\leonquidCorrect&\leonquidIncorrect&\leonquidError\\
tree\_inorder&5&0.123&4 (80\%)&(3,4)$^{}$&1+2 (60\%)&(1+3,1+4)$^{}$&\leonquidCorrect&\leonquidCorrect&\leonquidIncorrect&\leonquidError\\
tree\_map&7&0.061&4 (57\%)&\labelBlankHigherOrder&1+3 (57\%)&\labelBlankHigherOrder&\leonquidHigherOrderFunc&\leonquidHigherOrderFunc&\leonquidHigherOrderFunc&\leonquidHigherOrderFunc\\
tree\_nodes\_at\_level&11&\labelTimeout&\labelBlankOneFailed&\labelBlankOneFailed&\labelBlankOneFailed&\labelBlankOneFailed&\labelBlankOneFailed&\labelBlankOneFailed&\labelBlankOneFailed&\labelBlankOneFailed\\
tree\_postorder&20&\labelTimeout&\labelBlankOneFailed&\labelBlankOneFailed&\labelBlankOneFailed&\labelBlankOneFailed&\labelBlankOneFailed&\labelBlankOneFailed&\labelBlankOneFailed&\labelBlankOneFailed\\
tree\_preorder&5&0.153&3 (60\%)&(3,4)$^{}$$^{\labelRandomTime{3}}$&1+2 (60\%)&(1+3,1+3)$^{}$&\leonquidCorrect&\leonquidCorrect&\leonquidIncorrect&\leonquidIncorrect\\
&&&&&&&&&&\\

\hline
\textbf{Averages} &
&
&
\displayPctUpperBound{\pctFewerExamplesTopOne} &
&
\displayPct{\phantom{1+}\pctFewerExamplesBaseCaseStrategy} &
&&
\end{tabular}

\vsepBeforeCaption
  \captionsetup{justification=centering}
  \caption{
    Experiments.
      \textbf{Top-1(-R)}:
      1st (recursive) solution valid.
    \textbf{Time}:
       Average of 10 runs, in seconds.
    \\
    \textbf{2a Average}:
      \pctFewerExamplesTopOne{} for \numBenchmarks{} non-blank rows.
      (*Upper bound: \pctFewerExamplesTopOneUpperBound{} for all
     \numBenchmarksAll{} rows.)
    \\
    \textbf{3a Average}:
      \pctFewerExamplesBaseCaseStrategy{} for
      \numBenchmarksBase{} non-blank, non-error rows.
  }
\label{fig:experiments}
\end{figure}

\setcounter{figure}{0}

\begin{table}

\experimentTableSize

\begin{tabular}{l|cccccc}
& \multicolumn{6}{c}{\textbf{Experiment 1}} \\\hline
\textbf{Name} &
\textbf{Expert} & \textbf{Expert} & \textbf{Expert} &
\textbf{Time} & \textbf{Time} & \textbf{Time} \\
&
\textit{Submission:} & \multicolumn{2}{c}{\textit{Revised Artifact:}} &
\textit{Submission:} & \multicolumn{2}{c}{\textit{Revised Artifact:}} \\
&
\textit{Fig. 10} & \textit{Ours} & \textit{Yours} &
\textit{Fig. 10} & \textit{Ours} & \textit{Yours} \\
\\ 
bool\_band&4&4&\highlightRed{$\bullet$}&0.004&0.004&\highlightRed{$\bullet$}\\ 
bool\_bor&4&4&\highlightRed{$\bullet$}&0.004&0.004&\highlightRed{$\bullet$}\\ 
bool\_impl&4&4&\highlightRed{$\bullet$}&0.005&\highlightBlue{0.004}&\highlightRed{$\bullet$}\\ 
bool\_neg&2&2&\highlightRed{$\bullet$}&0.002&\highlightBlue{0.001}&\highlightRed{$\bullet$}\\ 
bool\_xor&4&4&\highlightRed{$\bullet$}&0.011&\highlightBlue{0.009}&\highlightRed{$\bullet$}\\ 
\\ 
list\_append&6&6&\highlightRed{$\bullet$}&0.008&0.008&\highlightRed{$\bullet$}\\ 
list\_compress&13&\highlightBlue{$\bullet$}&$\bullet$&timeout&\highlightBlue{$\bullet$}&$\bullet$\\ 
list\_concat&6&6&\highlightRed{$\bullet$}&0.008&\highlightBlue{0.009}&\highlightRed{$\bullet$}\\ 
list\_drop&11&11&\highlightRed{$\bullet$}&0.030&\highlightBlue{0.093}&\highlightRed{$\bullet$}\\ 
list\_even\_parity&7&\highlightBlue{$\bullet$}&$\bullet$&0.051&\highlightBlue{$\bullet$}&$\bullet$\\ 
list\_filter&8&8&\highlightRed{$\bullet$}&0.130&\highlightBlue{0.125}&\highlightRed{$\bullet$}\\ 
list\_fold&9&9&\highlightRed{$\bullet$}&0.765&\highlightBlue{0.852}&\highlightRed{$\bullet$}\\ 
list\_hd&3&3&\highlightRed{$\bullet$}&0.003&0.003&\highlightRed{$\bullet$}\\ 
list\_inc&4&4&\highlightRed{$\bullet$}&0.184&\highlightBlue{0.018}&\highlightRed{$\bullet$}\\ 
list\_last&6&6&\highlightRed{$\bullet$}&0.007&\highlightBlue{0.008}&\highlightRed{$\bullet$}\\ 
list\_length&3&3&\highlightRed{$\bullet$}&0.003&\highlightBlue{0.002}&\highlightRed{$\bullet$}\\ 
list\_map&8&8&\highlightRed{$\bullet$}&0.039&\highlightBlue{0.048}&\highlightRed{$\bullet$}\\ 
list\_nth&13&13&\highlightRed{$\bullet$}&0.113&\highlightBlue{0.131}&\highlightRed{$\bullet$}\\ 
list\_pairwise\_swap&7&7&\highlightRed{$\bullet$}&4.229&\highlightBlue{1.386}&\highlightRed{$\bullet$}\\ 
list\_rev\_append&5&5&\highlightRed{$\bullet$}&0.097&\highlightBlue{0.111}&\highlightRed{$\bullet$}\\ 
list\_rev\_fold&5&5&\highlightRed{$\bullet$}&0.027&\highlightBlue{0.035}&\highlightRed{$\bullet$}\\ 
list\_rev\_snoc&5&5&\highlightRed{$\bullet$}&0.009&\highlightBlue{0.010}&\highlightRed{$\bullet$}\\ 
list\_rev\_tailcall&8&8&\highlightRed{$\bullet$}&0.007&\highlightBlue{0.008}&\highlightRed{$\bullet$}\\ 
list\_snoc&8&8&\highlightRed{$\bullet$}&0.012&\highlightBlue{0.014}&\highlightRed{$\bullet$}\\ 
list\_sort\_sorted\_insert&7&7&\highlightRed{$\bullet$}&0.015&0.015&\highlightRed{$\bullet$}\\ 
list\_sorted\_insert&12&12&\highlightRed{$\bullet$}&10.964&\highlightBlue{3.034}&\highlightRed{$\bullet$}\\ 
list\_stutter&3&3&\highlightRed{$\bullet$}&0.004&\highlightBlue{0.003}&\highlightRed{$\bullet$}\\ 
list\_sum&3&3&\highlightRed{$\bullet$}&0.023&\highlightBlue{0.029}&\highlightRed{$\bullet$}\\ 
list\_take&12&12&\highlightRed{$\bullet$}&0.075&\highlightBlue{0.070}&\highlightRed{$\bullet$}\\ 
list\_tl&3&3&\highlightRed{$\bullet$}&0.003&\highlightBlue{0.002}&\highlightRed{$\bullet$}\\ 
\\ 
nat\_add&9&9&\highlightRed{$\bullet$}&0.007&\highlightBlue{0.006}&\highlightRed{$\bullet$}\\ 
nat\_iseven&4&4&\highlightRed{$\bullet$}&0.004&\highlightBlue{0.003}&\highlightRed{$\bullet$}\\ 
nat\_max&9&9&\highlightRed{$\bullet$}&0.039&\highlightBlue{0.043}&\highlightRed{$\bullet$}\\ 
nat\_pred&3&3&\highlightRed{$\bullet$}&0.002&\highlightBlue{0.001}&\highlightRed{$\bullet$}\\ 
\\ 
tree\_binsert&20&\highlightBlue{$\bullet$}&$\bullet$&timeout&\highlightBlue{$\bullet$}&$\bullet$\\ 
tree\_collect\_leaves&6&6&\highlightRed{$\bullet$}&0.066&\highlightBlue{0.079}&\highlightRed{$\bullet$}\\ 
tree\_count\_leaves&7&7&\highlightRed{$\bullet$}&3.009&\highlightBlue{2.822}&\highlightRed{$\bullet$}\\ 
tree\_count\_nodes&6&6&\highlightRed{$\bullet$}&0.323&\highlightBlue{0.361}&\highlightRed{$\bullet$}\\ 
tree\_inorder&5&5&\highlightRed{$\bullet$}&0.114&\highlightBlue{0.127}&\highlightRed{$\bullet$}\\ 
tree\_map&7&7&\highlightRed{$\bullet$}&0.055&\highlightBlue{0.065}&\highlightRed{$\bullet$}\\ 
tree\_nodes\_at\_level&11&\highlightBlue{$\bullet$}&$\bullet$&timeout&\highlightBlue{$\bullet$}&$\bullet$\\ 
tree\_postorder&20&\highlightBlue{$\bullet$}&$\bullet$&timeout&\highlightBlue{$\bullet$}&$\bullet$\\ 
tree\_preorder&5&5&\highlightRed{$\bullet$}&0.145&\highlightBlue{0.161}&\highlightRed{$\bullet$}\\ 

\end{tabular}

\vspace{0.10in}

\caption{Experiment 1.}

\end{table}

\begin{table}

\experimentTableSize

\begin{tabular}{l|cccccc}
& \multicolumn{3}{c}{\textbf{Experiment 2a}}
& \multicolumn{3}{c}{\textbf{Experiment 2b}} \\\hline
\textbf{Name} &
\textbf{Expert} & \textbf{Expert} & \textbf{Expert} &
\textbf{Random} & \textbf{Random} & \textbf{Random} \\
&
\textit{Submission:} & \multicolumn{2}{c}{\textit{Revised Artifact:}} &
\textit{Submission:} & \multicolumn{2}{c}{\textit{Revised Artifact:}} \\
&
\textit{Fig. 10} & \textit{Ours} & \textit{Yours} &
\textit{Fig. 10} & \textit{Ours} & \textit{Yours} \\
\\
bool\_band&3 (75\%)&3 (75\%)&\highlightRed{$\bullet$}&(4,4)$^{}$&(4,4)$^{}$&\highlightRed{$\bullet$}\\
bool\_bor&3 (75\%)&3 (75\%)&\highlightRed{$\bullet$}&(4,4)$^{}$&(4,4)$^{}$&\highlightRed{$\bullet$}\\
bool\_impl&3 (75\%)&3 (75\%)&\highlightRed{$\bullet$}&(3,4)$^{}$&\highlightBlue{(4,4)$^{}$}&\highlightRed{$\bullet$}\\
bool\_neg&2 (100\%)&2 (100\%)&\highlightRed{$\bullet$}&(2,2)$^{}$&(2,2)$^{}$&\highlightRed{$\bullet$}\\
bool\_xor&3 (75\%)&\highlightBlue{4 (100\%)}&\highlightRed{$\bullet$}&(4,4)$^{}$&(4,4)$^{}$&\highlightRed{$\bullet$}\\
\\
list\_append&4 (67\%)&4 (67\%)&\highlightRed{$\bullet$}&(3,5)$^{}$&\highlightBlue{(3,4)$^{}$}&\highlightRed{$\bullet$}\\
list\_compress&---&\highlightBlue{$\bullet$}&$\bullet$&---&\highlightBlue{$\bullet$}&$\bullet$\\
list\_concat&3 (50\%)&3 (50\%)&\highlightRed{$\bullet$}&(2,3)$^{}$&\highlightBlue{(3,4)$^{}$}&\highlightRed{$\bullet$}\\
list\_drop&5 (45\%)&5 (45\%)&\highlightRed{$\bullet$}&(6,9)$^{}$&(6,9)$^{}$&\highlightRed{$\bullet$}\\
list\_even\_parity&5 (71\%)&\highlightBlue{$\bullet$}&$\bullet$&\scriptsize{failed}&\highlightBlue{$\bullet$}&$\bullet$\\
list\_filter&4 (50\%)&\highlightBlue{$\bullet$}&$\bullet$&---&\highlightBlue{$\bullet$}&$\bullet$\\
list\_fold&3 (33\%)&3 (33\%)&\highlightRed{$\bullet$}&---&\highlightBlue{$\bullet$}&$\bullet$\\
list\_hd&2 (67\%)&2 (67\%)&\highlightRed{$\bullet$}&(2,3)$^{}$&(2,3)$^{}$&\highlightRed{$\bullet$}\\
list\_inc&2 (50\%)&2 (50\%)&\highlightRed{$\bullet$}&(2,2)$^{}$&(2,2)$^{}$&\highlightRed{$\bullet$}\\
list\_last&4 (67\%)&4 (67\%)&\highlightRed{$\bullet$}&(5,12)$^{}$&\highlightBlue{(5,9)$^{}$}&\highlightRed{$\bullet$}\\
list\_length&3 (100\%)&3 (100\%)&\highlightRed{$\bullet$}&(2,3)$^{}$&\highlightBlue{(3,4)$^{}$}&\highlightRed{$\bullet$}\\
list\_map&4 (50\%)&4 (50\%)&\highlightRed{$\bullet$}&---&\highlightBlue{$\bullet$}&$\bullet$\\
list\_nth&5 (38\%)&5 (38\%)&\highlightRed{$\bullet$}&(8,15)$^{}$&\highlightBlue{(8,12)$^{}$}&\highlightRed{$\bullet$}\\
list\_pairwise\_swap&5 (71\%)&5 (71\%)&\highlightRed{$\bullet$}&\scriptsize{timeout}&\highlightBlue{$\bullet$}&$\bullet$\\
list\_rev\_append&3 (60\%)&3 (60\%)&\highlightRed{$\bullet$}&(5,9)$^{}$&(5,9)$^{}$&\highlightRed{$\bullet$}\\
list\_rev\_fold&2 (40\%)&2 (40\%)&\highlightRed{$\bullet$}&(2,3)$^{}$&\highlightBlue{(3,3)$^{}$}&\highlightRed{$\bullet$}\\
list\_rev\_snoc&3 (60\%)&3 (60\%)&\highlightRed{$\bullet$}&(3,7)$^{}$&\highlightBlue{(3,6)$^{}$}&\highlightRed{$\bullet$}\\
list\_rev\_tailcall&3 (38\%)&3 (38\%)&\highlightRed{$\bullet$}&(3,6)$^{}$&\highlightBlue{(3,5)$^{}$}&\highlightRed{$\bullet$}\\
list\_snoc&4 (50\%)&4 (50\%)&\highlightRed{$\bullet$}&(3,4)$^{}$&(3,4)$^{}$&\highlightRed{$\bullet$}\\
list\_sort\_sorted\_insert&3 (43\%)&3 (43\%)&\highlightRed{$\bullet$}&(3,6)$^{}$&\highlightBlue{(3,5)$^{}$}&\highlightRed{$\bullet$}\\
list\_sorted\_insert&7 (58\%)&7 (58\%)&\highlightRed{$\bullet$}&\scriptsize{timeout}&\highlightBlue{$\bullet$}&$\bullet$\\
list\_stutter&2 (67\%)&2 (67\%)&\highlightRed{$\bullet$}&(3,3)$^{}$&\highlightBlue{(3,4)$^{}$}&\highlightRed{$\bullet$}\\
list\_sum&2 (67\%)&2 (67\%)&\highlightRed{$\bullet$}&(2,3)$^{}$&\highlightBlue{(2,2)$^{}$}&\highlightRed{$\bullet$}\\
list\_take&6 (50\%)&\highlightBlue{5 (42\%)}&\highlightRed{$\bullet$}&(7,10)$^{}$&\highlightBlue{(7,9)$^{}$}&\highlightRed{$\bullet$}\\
list\_tl&2 (67\%)&2 (67\%)&\highlightRed{$\bullet$}&(2,3)$^{}$&(2,3)$^{}$&\highlightRed{$\bullet$}\\
\\
nat\_add&4 (44\%)&4 (44\%)&\highlightRed{$\bullet$}&(4,6)$^{}$&\highlightBlue{(5,6)$^{}$}&\highlightRed{$\bullet$}\\
nat\_iseven&3 (75\%)&3 (75\%)&\highlightRed{$\bullet$}&(3,4)$^{}$&\highlightBlue{(4,4)$^{}$}&\highlightRed{$\bullet$}\\
nat\_max&9 (100\%)&9 (100\%)&\highlightRed{$\bullet$}&(9,11)$^{}$&\highlightBlue{(9,12)$^{}$}&\highlightRed{$\bullet$}\\
nat\_pred&2 (67\%)&2 (67\%)&\highlightRed{$\bullet$}&(2,3)$^{}$&(2,3)$^{}$&\highlightRed{$\bullet$}\\
\\
tree\_binsert&---&\highlightBlue{$\bullet$}&$\bullet$&---&\highlightBlue{$\bullet$}&$\bullet$\\
tree\_collect\_leaves&3 (50\%)&3 (50\%)&\highlightRed{$\bullet$}&(3,4)$^{3}$&\highlightBlue{(3,4)$^{}$}&\highlightRed{$\bullet$}\\
tree\_count\_leaves&3 (43\%)&3 (43\%)&\highlightRed{$\bullet$}&\scriptsize{timeout}&\highlightBlue{$\bullet$}&$\bullet$\\
tree\_count\_nodes&3 (50\%)&3 (50\%)&\highlightRed{$\bullet$}&(4,$\downarrow$)$^{10}$&\highlightBlue{(4,---)$^{}$}&\highlightRed{$\bullet$}\\
tree\_inorder&4 (80\%)&4 (80\%)&\highlightRed{$\bullet$}&(3,4)$^{}$&(3,4)$^{}$&\highlightRed{$\bullet$}\\
tree\_map&4 (57\%)&4 (57\%)&\highlightRed{$\bullet$}&---&\highlightBlue{$\bullet$}&$\bullet$\\
tree\_nodes\_at\_level&---&\highlightBlue{$\bullet$}&$\bullet$&---&\highlightBlue{$\bullet$}&$\bullet$\\
tree\_postorder&---&\highlightBlue{$\bullet$}&$\bullet$&---&\highlightBlue{$\bullet$}&$\bullet$\\
tree\_preorder&3 (60\%)&3 (60\%)&\highlightRed{$\bullet$}&(3,3)$^{3}$&\highlightBlue{(3,3)$^{}$}&\highlightRed{$\bullet$}\\

\end{tabular}

\vspace{0.10in}

\caption{Experiment 2.
%
Differences (in blue) between results from \snsMyth{} at submission and
\snsMyth{} now:
%
\experimentCaptionSize
%
\\[3pt]
%
\textbf{list\_compress, tree\_binsert, tree\_nodes\_at\_level:} Not run because
they failed in Experiment 1.
%
\\[3pt]
%
\textbf{Expert: bool\_xor:} With the algorithmic changes, \snsMyth{} now
requires (all) 4 examples.  (Small changes to search order and search parameters
can change the results of synthesis tools.)
%
\\[3pt]
%
\textbf{Expert: list\_snoc, list\_take:} When looking through our tasks again,
we noticed an opportunity to try removing another example from these benchmarks;
\snsMyth{} produces correct solutions given the fewer examples.
%
\\[3pt]
%
\textbf{Random:} Small variations in k50 and k90 are expected because the
examples are generated randomly. There are some blue dots because our scripts
for benchmarking and generating the table differences do not automatically
display {\scriptsize{failed}}, {\scriptsize{timeout}}, superscripts 3 and 10, or
the $\downarrow$ arrow presented in Figure 10.
%
\\[3pt]
%
}

\end{table}

\begin{table}

\experimentTableSize

\begin{tabular}{l|cccccc}
& \multicolumn{3}{c}{\textbf{Experiment 3a}}
& \multicolumn{3}{c}{\textbf{Experiment 3b}} \\\hline
\textbf{Name} &
\textbf{Expert} & \textbf{Expert} & \textbf{Expert} &
\textbf{Random} & \textbf{Random} & \textbf{Random} \\
&
\textit{Submission:} & \multicolumn{2}{c}{\textit{Revised Artifact:}} &
\textit{Submission:} & \multicolumn{2}{c}{\textit{Revised Artifact:}} \\
&
\textit{Fig. 10} & \textit{Ours} & \textit{Yours} &
\textit{Fig. 10} & \textit{Ours} & \textit{Yours} \\
\\ 
bool\_band&---&\highlightBlue{$\bullet$}&$\bullet$&---&\highlightBlue{$\bullet$}&$\bullet$\\ 
bool\_bor&---&\highlightBlue{$\bullet$}&$\bullet$&---&\highlightBlue{$\bullet$}&$\bullet$\\ 
bool\_impl&---&\highlightBlue{$\bullet$}&$\bullet$&---&\highlightBlue{$\bullet$}&$\bullet$\\ 
bool\_neg&---&\highlightBlue{$\bullet$}&$\bullet$&---&\highlightBlue{$\bullet$}&$\bullet$\\ 
bool\_xor&---&\highlightBlue{$\bullet$}&$\bullet$&---&\highlightBlue{$\bullet$}&$\bullet$\\ 
\\ 
list\_append&1+1 (33\%)&1+1 (33\%)&\highlightRed{$\bullet$}&(1+3,1+5)$^{}$&\highlightBlue{(1+3,1+4)$^{}$}&\highlightRed{$\bullet$}\\ 
list\_compress&---&\highlightBlue{$\bullet$}&$\bullet$&---&\highlightBlue{$\bullet$}&$\bullet$\\ 
list\_concat&1+2 (50\%)&\highlightBlue{1+1 (33\%)}&\highlightRed{$\bullet$}&(1+2,1+3)$^{}$&(1+2,1+3)$^{}$&\highlightRed{$\bullet$}\\ 
list\_drop&1+2 (27\%)&1+2 (27\%)&\highlightRed{$\bullet$}&(1+8,1+15)$^{}$&\highlightBlue{(1+7,1+16)$^{}$}&\highlightRed{$\bullet$}\\ 
list\_even\_parity&\scriptsize{overspec}&\highlightBlue{$\bullet$}&$\bullet$&\scriptsize{failed}&\highlightBlue{$\bullet$}&$\bullet$\\ 
list\_filter&\scriptsize{overspec}&\highlightBlue{1+3 (50\%)}&\highlightRed{$\bullet$}&---&\highlightBlue{$\bullet$}&$\bullet$\\ 
list\_fold&1+3 (44\%)&1+3 (44\%)&\highlightRed{$\bullet$}&---&\highlightBlue{$\bullet$}&$\bullet$\\ 
list\_hd&---&\highlightBlue{$\bullet$}&$\bullet$&---&\highlightBlue{$\bullet$}&$\bullet$\\ 
list\_inc&---&\highlightBlue{$\bullet$}&$\bullet$&---&\highlightBlue{$\bullet$}&$\bullet$\\ 
list\_last&1+2 (50\%)&1+2 (50\%)&\highlightRed{$\bullet$}&(1+5,1+8)$^{}$&\highlightBlue{(1+4,1+8)$^{}$}&\highlightRed{$\bullet$}\\ 
list\_length&1+1 (67\%)&1+1 (67\%)&\highlightRed{$\bullet$}&(1+2,1+3)$^{}$&\highlightBlue{(1+2,1+2)$^{}$}&\highlightRed{$\bullet$}\\ 
list\_map&1+2 (38\%)&1+2 (38\%)&\highlightRed{$\bullet$}&---&\highlightBlue{$\bullet$}&$\bullet$\\ 
list\_nth&1+2 (23\%)&1+2 (23\%)&\highlightRed{$\bullet$}&(1+8,1+16)$^{}$&\highlightBlue{(1+6,1+14)$^{}$}&\highlightRed{$\bullet$}\\ 
list\_pairwise\_swap&\scriptsize{overspec}&\highlightBlue{$\bullet$}&$\bullet$&\scriptsize{timeout}&\highlightBlue{$\bullet$}&$\bullet$\\ 
list\_rev\_append&1+2 (60\%)&1+2 (60\%)&\highlightRed{$\bullet$}&(1+3,1+14)$^{}$&\highlightBlue{(1+3,1+4)$^{}$}&\highlightRed{$\bullet$}\\ 
list\_rev\_fold&---&\highlightBlue{$\bullet$}&$\bullet$&---&\highlightBlue{$\bullet$}&$\bullet$\\ 
list\_rev\_snoc&1+1 (40\%)&1+1 (40\%)&\highlightRed{$\bullet$}&(1+3,1+5)$^{}$&\highlightBlue{(1+3,1+4)$^{}$}&\highlightRed{$\bullet$}\\ 
list\_rev\_tailcall&1+1 (25\%)&1+1 (25\%)&\highlightRed{$\bullet$}&(1+3,1+6)$^{}$&\highlightBlue{(1+3,1+4)$^{}$}&\highlightRed{$\bullet$}\\ 
list\_snoc&1+2 (38\%)&\highlightBlue{1+1 (25\%)}&\highlightRed{$\bullet$}&(1+2,1+4)$^{}$&(1+2,1+4)$^{}$&\highlightRed{$\bullet$}\\ 
list\_sort\_sorted\_insert&1+1 (29\%)&1+1 (29\%)&\highlightRed{$\bullet$}&(1+3,1+6)$^{}$&\highlightBlue{(1+2,1+4)$^{}$}&\highlightRed{$\bullet$}\\ 
list\_sorted\_insert&\scriptsize{overspec}&\highlightBlue{1+7 (67\%)}&\highlightRed{$\bullet$}&\scriptsize{timeout}&\highlightBlue{$\bullet$}&$\bullet$\\ 
list\_stutter&1+1 (67\%)&1+1 (67\%)&\highlightRed{$\bullet$}&(1+3,1+4)$^{}$&\highlightBlue{(1+2,1+3)$^{}$}&\highlightRed{$\bullet$}\\ 
list\_sum&---&\highlightBlue{$\bullet$}&$\bullet$&---&\highlightBlue{$\bullet$}&$\bullet$\\ 
list\_take&1+3 (33\%)&1+3 (33\%)&\highlightRed{$\bullet$}&(1+8,1+15)$^{}$&\highlightBlue{(1+7,1+13)$^{}$}&\highlightRed{$\bullet$}\\ 
list\_tl&---&\highlightBlue{$\bullet$}&$\bullet$&---&\highlightBlue{$\bullet$}&$\bullet$\\ 
\\ 
nat\_add&1+1 (22\%)&1+1 (22\%)&\highlightRed{$\bullet$}&(1+4,1+6)$^{}$&\highlightBlue{(1+3,1+5)$^{}$}&\highlightRed{$\bullet$}\\ 
nat\_iseven&1+2 (75\%)&1+2 (75\%)&\highlightRed{$\bullet$}&(1+3,1+4)$^{}$&(1+3,1+4)$^{}$&\highlightRed{$\bullet$}\\ 
nat\_max&1+4 (56\%)&1+4 (56\%)&\highlightRed{$\bullet$}&(1+8,1+12)$^{}$&\highlightBlue{(1+9,1+12)$^{}$}&\highlightRed{$\bullet$}\\ 
nat\_pred&---&\highlightBlue{$\bullet$}&$\bullet$&---&\highlightBlue{$\bullet$}&$\bullet$\\ 
\\ 
tree\_binsert&---&\highlightBlue{$\bullet$}&$\bullet$&---&\highlightBlue{$\bullet$}&$\bullet$\\ 
tree\_collect\_leaves&1+2 (50\%)&1+2 (50\%)&\highlightRed{$\bullet$}&(1+3,1+3)$^{}$&(1+3,1+3)$^{}$&\highlightRed{$\bullet$}\\ 
tree\_count\_leaves&1+1 (29\%)&1+1 (29\%)&\highlightRed{$\bullet$}&\scriptsize{timeout}&\highlightBlue{$\bullet$}&$\bullet$\\ 
tree\_count\_nodes&1+2 (50\%)&1+2 (50\%)&\highlightRed{$\bullet$}&(1+4,1+5)$^{10}$&\highlightBlue{(1+3,1+4)$^{}$}&\highlightRed{$\bullet$}\\ 
tree\_inorder&1+2 (60\%)&1+2 (60\%)&\highlightRed{$\bullet$}&(1+3,1+3)$^{}$&(1+3,1+3)$^{}$&\highlightRed{$\bullet$}\\ 
tree\_map&1+3 (57\%)&1+3 (57\%)&\highlightRed{$\bullet$}&---&\highlightBlue{$\bullet$}&$\bullet$\\ 
tree\_nodes\_at\_level&---&\highlightBlue{$\bullet$}&$\bullet$&---&\highlightBlue{$\bullet$}&$\bullet$\\ 
tree\_postorder&---&\highlightBlue{$\bullet$}&$\bullet$&---&\highlightBlue{$\bullet$}&$\bullet$\\ 
tree\_preorder&1+2 (60\%)&1+2 (60\%)&\highlightRed{$\bullet$}&(1+3,1+3)$^{}$&(1+3,1+3)$^{}$&\highlightRed{$\bullet$}\\ 

\end{tabular}

\vspace{0.10in}

\caption{Experiment 3.}

\end{table}


\end{document}
\endinput
